\documentclass{article}
\usepackage{fullpage,amsmath,amssymb,proof,url}

\newcommand{\cd}[1]{\texttt{#1}}
\newcommand{\mt}[1]{\mathsf{#1}}

\newcommand{\rc}{+ \hspace{-.075in} + \;}
\newcommand{\rcut}{\; \texttt{--} \;}
\newcommand{\rcutM}{\; \texttt{---} \;}

\begin{document}

\title{The Ur/Web Manual}
\author{Adam Chlipala}

\maketitle

\tableofcontents


\section{Introduction}

\emph{Ur} is a programming language designed to introduce richer type system features into functional programming in the tradition of ML and Haskell.  Ur is functional, pure, statically-typed, and strict.  Ur supports a powerful kind of \emph{metaprogramming} based on \emph{row types}.

\emph{Ur/Web} is Ur plus a special standard library and associated rules for parsing and optimization.  Ur/Web supports construction of dynamic web applications backed by SQL databases.  The signature of the standard library is such that well-typed Ur/Web programs ``don't go wrong'' in a very broad sense.  Not only do they not crash during particular page generations, but they also may not:

\begin{itemize}
\item Suffer from any kinds of code-injection attacks
\item Return invalid HTML
\item Contain dead intra-application links
\item Have mismatches between HTML forms and the fields expected by their handlers
\item Attempt invalid SQL queries
\item Use improper marshaling or unmarshaling in communication with SQL databases
\end{itemize}

This type safety is just the foundation of the Ur/Web methodology.  It is also possible to use metaprogramming to build significant application pieces by analysis of type structure.  For instance, the demo includes an ML-style functor for building an admin interface for an arbitrary SQL table.  The type system guarantees that the admin interface sub-application that comes out will always be free of the above-listed bugs, no matter which well-typed table description is given as input.

The Ur/Web compiler also produces very efficient object code that does not use garbage collection.  These compiled programs will often be even more efficient than what most programmers would bother to write in C.

\medskip

The official web site for Ur is:
\begin{center}
  \url{http://www.impredicative.com/ur/}
\end{center}


\section{Installation}

If you are lucky, then the following standard command sequence will suffice for installation, in a directory to which you have unpacked the latest distribution tarball.

\begin{verbatim}
./configure
make
sudo make install
\end{verbatim}

Some other packages must be installed for the above to work.  At a minimum, you need a standard UNIX shell, with standard UNIX tools like sed and GCC in your execution path; and MLton, the whole-program optimizing compiler for Standard ML.  To build programs that access SQL databases, you also need libpq, the PostgreSQL client library.  As of this writing, in the ``testing'' version of Debian Linux, this command will install the more uncommon of these dependencies:

\begin{verbatim}
apt-get install mlton libpq-dev
\end{verbatim}

It is also possible to access the modules of the Ur/Web compiler interactively, within Standard ML of New Jersey.  To install the prerequisites in Debian testing:

\begin{verbatim}
apt-get install smlnj libsmlnj-smlnj ml-yacc ml-lpt
\end{verbatim}

To begin an interactive session with the Ur compiler modules, run \texttt{make smlnj}, and then, from within an \texttt{sml} session, run \texttt{CM.make "src/urweb.cm";}.  The \texttt{Compiler} module is the main entry point.

To run an SQL-backed application, you will probably want to install the PostgreSQL server.  Version 8.3 or higher is required.

\begin{verbatim}
apt-get install postgresql-8.3
\end{verbatim}

To use the Emacs mode, you must have a modern Emacs installed.  We assume that you already know how to do this, if you're in the business of looking for an Emacs mode.  The demo generation facility of the compiler will also call out to Emacs to syntax-highlight code, and that process depends on the \texttt{htmlize} module, which can be installed in Debian testing via:

\begin{verbatim}
apt-get install emacs-goodies-el
\end{verbatim}

Even with the right packages installed, configuration and building might fail to work.  After you run \texttt{./configure}, you will see the values of some named environment variables printed.  You may need to adjust these values to get proper installation for your system.  To change a value, store your preferred alternative in the corresponding UNIX environment variable, before running \texttt{./configure}.  For instance, here is how to change the list of extra arguments that the Ur/Web compiler will pass to GCC on every invocation.

\begin{verbatim}
GCCARGS=-fnested-functions ./configure
\end{verbatim}

Some OSX users have reported needing to use this particular GCCARGS value.

The Emacs mode can be set to autoload by adding the following to your \texttt{.emacs} file.

\begin{verbatim}
(add-to-list 'load-path "/usr/local/share/emacs/site-lisp/urweb-mode")
(load "urweb-mode-startup")
\end{verbatim}

Change the path in the first line if you chose a different Emacs installation path during configuration.


\section{Command-Line Compiler}

\subsection{Project Files}

The basic inputs to the \texttt{urweb} compiler are project files, which have the extension \texttt{.urp}.  Here is a sample \texttt{.urp} file.

\begin{verbatim}
database dbname=test
sql crud1.sql

crud
crud1
\end{verbatim}

The \texttt{database} line gives the database information string to pass to libpq.  In this case, the string only says to connect to a local database named \texttt{test}.

The \texttt{sql} line asks for an SQL source file to be generated, giving the commands to run to create the tables and sequences that this application expects to find.  After building this \texttt{.urp} file, the following commands could be used to initialize the database, assuming that the current UNIX user exists as a Postgres user with database creation privileges:

\begin{verbatim}
createdb test
psql -f crud1.sql test
\end{verbatim}

A blank line always separates the named directives from a list of modules to include in the project; if there are no named directives, a blank line must begin the file.

For each entry \texttt{M} in the module list, the file \texttt{M.urs} is included in the project if it exists, and the file \texttt{M.ur} must exist and is always included.

A few other named directives are supported.  \texttt{prefix PREFIX} sets the prefix included before every URI within the generated application; the default is \texttt{/}.  \texttt{exe FILENAME} sets the filename to which to write the output executable; the default for file \texttt{P.urp} is \texttt{P.exe}.  \texttt{debug} saves some intermediate C files, which is mostly useful to help in debugging the compiler itself.  \texttt{profile} generates an executable that may be used with gprof.

\subsection{Building an Application}

To compile project \texttt{P.urp}, simply run
\begin{verbatim}
urweb P
\end{verbatim}
The output executable is a standalone web server.  Run it with the command-line argument \texttt{-h} to see which options it takes.  If the project file lists a database, the web server will attempt to connect to that database on startup.

To time how long the different compiler phases run, without generating an executable, run
\begin{verbatim}
urweb -timing P
\end{verbatim}


\section{Ur Syntax}

In this section, we describe the syntax of Ur, deferring to a later section discussion of most of the syntax specific to SQL and XML.  The sole exceptions are the declaration forms for tables, sequences, and cookies.

\subsection{Lexical Conventions}

We give the Ur language definition in \LaTeX $\;$ math mode, since that is prettier than monospaced ASCII.  The corresponding ASCII syntax can be read off directly.  Here is the key for mapping math symbols to ASCII character sequences.

\begin{center}
  \begin{tabular}{rl}
    \textbf{\LaTeX} & \textbf{ASCII} \\
    $\to$ & \cd{->} \\
    $\times$ & \cd{*} \\
    $\lambda$ & \cd{fn} \\
    $\Rightarrow$ & \cd{=>} \\
    $\neq$ & \cd{<>} \\
    $\leq$ & \cd{<=} \\
    $\geq$ & \cd{>=} \\
    \\
    $x$ & Normal textual identifier, not beginning with an uppercase letter \\
    $X$ & Normal textual identifier, beginning with an uppercase letter \\
  \end{tabular}
\end{center}

We often write syntax like $e^*$ to indicate zero or more copies of $e$, $e^+$ to indicate one or more copies, and $e,^*$ and $e,^+$ to indicate multiple copies separated by commas.  Another separator may be used in place of a comma.  The $e$ term may be surrounded by parentheses to indicate grouping; those parentheses should not be included in the actual ASCII.

We write $\ell$ for literals of the primitive types, for the most part following C conventions.  There are $\mt{int}$, $\mt{float}$, and $\mt{string}$ literals.

This version of the manual doesn't include operator precedences; see \texttt{src/urweb.grm} for that.

\subsection{\label{core}Core Syntax}

\emph{Kinds} classify types and other compile-time-only entities.  Each kind in the grammar is listed with a description of the sort of data it classifies.
$$\begin{array}{rrcll}
  \textrm{Kinds} & \kappa &::=& \mt{Type} & \textrm{proper types} \\
  &&& \mt{Unit} & \textrm{the trivial constructor} \\
  &&& \mt{Name} & \textrm{field names} \\
  &&& \kappa \to \kappa & \textrm{type-level functions} \\
  &&& \{\kappa\} & \textrm{type-level records} \\
  &&& (\kappa\times^+) & \textrm{type-level tuples} \\
  &&& \_\_ & \textrm{wildcard} \\
  &&& (\kappa) & \textrm{explicit precedence} \\
\end{array}$$

Ur supports several different notions of functions that take types as arguments.  These arguments can be either implicit, causing them to be inferred at use sites; or explicit, forcing them to be specified manually at use sites.  There is a common explicitness annotation convention applied at the definitions of and in the types of such functions.
$$\begin{array}{rrcll}
  \textrm{Explicitness} & ? &::=& :: & \textrm{explicit} \\
  &&& ::: & \textrm{implicit}
\end{array}$$

\emph{Constructors} are the main class of compile-time-only data.  They include proper types and are classified by kinds.
$$\begin{array}{rrcll}
  \textrm{Constructors} & c, \tau &::=& (c) :: \kappa & \textrm{kind annotation} \\
  &&& \hat{x} & \textrm{constructor variable} \\
  \\
  &&& \tau \to \tau & \textrm{function type} \\
  &&& x \; ? \; \kappa \to \tau & \textrm{polymorphic function type} \\
  &&& \$ c & \textrm{record type} \\
  \\
  &&& c \; c & \textrm{type-level function application} \\
  &&& \lambda x \; :: \; \kappa \Rightarrow c & \textrm{type-level function abstraction} \\
  \\
  &&& () & \textrm{type-level unit} \\
  &&& \#X & \textrm{field name} \\
  \\
  &&& [(c = c)^*] & \textrm{known-length type-level record} \\
  &&& c \rc c & \textrm{type-level record concatenation} \\
  &&& \mt{fold} & \textrm{type-level record fold} \\
  \\
  &&& (c,^+) & \textrm{type-level tuple} \\
  &&& c.n & \textrm{type-level tuple projection ($n \in \mathbb N^+$)} \\
  \\
  &&& \lambda [c \sim c] \Rightarrow c & \textrm{guarded constructor} \\
  \\
  &&& \_ :: \kappa & \textrm{wildcard} \\
  &&& (c) & \textrm{explicit precedence} \\
  \\
  \textrm{Qualified uncapitalized variables} & \hat{x} &::=& x & \textrm{not from a module} \\
  &&& M.x & \textrm{projection from a module} \\
\end{array}$$

Modules of the module system are described by \emph{signatures}.
$$\begin{array}{rrcll}
  \textrm{Signatures} & S &::=& \mt{sig} \; s^* \; \mt{end} & \textrm{constant} \\
  &&& X & \textrm{variable} \\
  &&& \mt{functor}(X : S) : S & \textrm{functor} \\
  &&& S \; \mt{where} \; \mt{con} \; x = c & \textrm{concretizing an abstract constructor} \\
  &&& M.X & \textrm{projection from a module} \\
  \\
  \textrm{Signature items} & s &::=& \mt{con} \; x :: \kappa & \textrm{abstract constructor} \\
  &&& \mt{con} \; x :: \kappa = c & \textrm{concrete constructor} \\
  &&& \mt{datatype} \; x \; x^* = dc\mid^+ & \textrm{algebraic datatype definition} \\
  &&& \mt{datatype} \; x = \mt{datatype} \; M.x & \textrm{algebraic datatype import} \\
  &&& \mt{val} \; x : \tau & \textrm{value} \\
  &&& \mt{structure} \; X : S & \textrm{sub-module} \\
  &&& \mt{signature} \; X = S & \textrm{sub-signature} \\
  &&& \mt{include} \; S & \textrm{signature inclusion} \\
  &&& \mt{constraint} \; c \sim c & \textrm{record disjointness constraint} \\
  &&& \mt{class} \; x & \textrm{abstract type class} \\
  &&& \mt{class} \; x = c & \textrm{concrete type class} \\
  \\
  \textrm{Datatype constructors} & dc &::=& X & \textrm{nullary constructor} \\
  &&& X \; \mt{of} \; \tau & \textrm{unary constructor} \\
\end{array}$$

\emph{Patterns} are used to describe structural conditions on expressions, such that expressions may be tested against patterns, generating assignments to pattern variables if successful.
$$\begin{array}{rrcll}
  \textrm{Patterns} & p &::=& \_ & \textrm{wildcard} \\
  &&& x & \textrm{variable} \\
  &&& \ell & \textrm{constant} \\
  &&& \hat{X} & \textrm{nullary constructor} \\
  &&& \hat{X} \; p & \textrm{unary constructor} \\
  &&& \{(x = p,)^*\} & \textrm{rigid record pattern} \\
  &&& \{(x = p,)^+, \ldots\} & \textrm{flexible record pattern} \\
  &&& (p) & \textrm{explicit precedence} \\
  \\
  \textrm{Qualified capitalized variables} & \hat{X} &::=& X & \textrm{not from a module} \\
  &&& M.X & \textrm{projection from a module} \\
\end{array}$$

\emph{Expressions} are the main run-time entities, corresponding to both ``expressions'' and ``statements'' in mainstream imperative languages.
$$\begin{array}{rrcll}
  \textrm{Expressions} & e &::=& e : \tau & \textrm{type annotation} \\
  &&& \hat{x} & \textrm{variable} \\
  &&& \hat{X} & \textrm{datatype constructor} \\
  &&& \ell & \textrm{constant} \\
  \\
  &&& e \; e & \textrm{function application} \\
  &&& \lambda x : \tau \Rightarrow e & \textrm{function abstraction} \\
  &&& e [c] & \textrm{polymorphic function application} \\
  &&& \lambda x \; ? \; \kappa \Rightarrow e & \textrm{polymorphic function abstraction} \\
  \\
  &&& \{(c = e,)^*\} & \textrm{known-length record} \\
  &&& e.c & \textrm{record field projection} \\
  &&& e \rc e & \textrm{record concatenation} \\
  &&& e \rcut c & \textrm{removal of a single record field} \\
  &&& e \rcutM c & \textrm{removal of multiple record fields} \\
  &&& \mt{fold} & \textrm{fold over fields of a type-level record} \\
  \\
  &&& \mt{let} \; ed^* \; \mt{in} \; e \; \mt{end} & \textrm{local definitions} \\
  \\
  &&& \mt{case} \; e \; \mt{of} \; (p \Rightarrow e|)^+ & \textrm{pattern matching} \\
  \\
  &&& \lambda [c \sim c] \Rightarrow e & \textrm{guarded expression} \\
  \\
  &&& \_ & \textrm{wildcard} \\
  &&& (e) & \textrm{explicit precedence} \\
  \\
  \textrm{Local declarations} & ed &::=& \cd{val} \; x : \tau = e & \textrm{non-recursive value} \\
  &&& \cd{val} \; \cd{rec} \; (x : \tau = e \; \cd{and})^+ & \textrm{mutually-recursive values} \\
\end{array}$$

\emph{Declarations} primarily bring new symbols into context.
$$\begin{array}{rrcll}
  \textrm{Declarations} & d &::=& \mt{con} \; x :: \kappa = c & \textrm{constructor synonym} \\
  &&& \mt{datatype} \; x \; x^* = dc\mid^+ & \textrm{algebraic datatype definition} \\
  &&& \mt{datatype} \; x = \mt{datatype} \; M.x & \textrm{algebraic datatype import} \\
  &&& \mt{val} \; x : \tau = e & \textrm{value} \\
  &&& \mt{val} \; \cd{rec} \; (x : \tau = e \; \mt{and})^+ & \textrm{mutually-recursive values} \\
  &&& \mt{structure} \; X : S = M & \textrm{module definition} \\
  &&& \mt{signature} \; X = S & \textrm{signature definition} \\
  &&& \mt{open} \; M & \textrm{module inclusion} \\
  &&& \mt{constraint} \; c \sim c & \textrm{record disjointness constraint} \\
  &&& \mt{open} \; \mt{constraints} \; M & \textrm{inclusion of just the constraints from a module} \\
  &&& \mt{table} \; x : c & \textrm{SQL table} \\
  &&& \mt{sequence} \; x & \textrm{SQL sequence} \\
  &&& \mt{cookie} \; x : \tau & \textrm{HTTP cookie} \\
  &&& \mt{class} \; x = c & \textrm{concrete type class} \\
  \\
  \textrm{Modules} & M &::=& \mt{struct} \; d^* \; \mt{end} & \textrm{constant} \\
  &&& X & \textrm{variable} \\
  &&& M.X & \textrm{projection} \\
  &&& M(M) & \textrm{functor application} \\
  &&& \mt{functor}(X : S) : S = M & \textrm{functor abstraction} \\
\end{array}$$

There are two kinds of Ur files.  A file named $M\texttt{.ur}$ is an \emph{implementation file}, and it should contain a sequence of declarations $d^*$.  A file named $M\texttt{.urs}$ is an \emph{interface file}; it must always have a matching $M\texttt{.ur}$ and should contain a sequence of signature items $s^*$.  When both files are present, the overall effect is the same as a monolithic declaration $\mt{structure} \; M : \mt{sig} \; s^* \; \mt{end} = \mt{struct} \; d^* \; \mt{end}$.  When no interface file is included, the overall effect is similar, with a signature for module $M$ being inferred rather than just checked against an interface.

\subsection{Shorthands}

There are a variety of derived syntactic forms that elaborate into the core syntax from the last subsection.  We will present the additional forms roughly following the order in which we presented the constructs that they elaborate into.

In many contexts where record fields are expected, like in a projection $e.c$, a constant field may be written as simply $X$, rather than $\#X$.

A record type may be written $\{(c = c,)^*\}$, which elaborates to $\$[(c = c,)^*]$.

The notation $[c_1, \ldots, c_n]$ is shorthand for $[c_1 = (), \ldots, c_n = ()]$.

A tuple type $(\tau_1, \ldots, \tau_n)$ expands to a record type $\{1 = \tau_1, \ldots, n = \tau_n\}$, with natural numbers as field names.  A tuple pattern $(p_1, \ldots, p_n)$ expands to a rigid record pattern $\{1 = p_1, \ldots, n = p_n\}$.  Positive natural numbers may be used in most places where field names would be allowed.

In general, several adjacent $\lambda$ forms may be combined into one, and kind and type annotations may be omitted, in which case they are implicitly included as wildcards.  More formally, for constructor-level abstractions, we can define a new non-terminal $b ::= x \mid (x :: \kappa) \mid [c \sim c]$ and allow composite abstractions of the form $\lambda b^+ \Rightarrow c$, elaborating into the obvious sequence of one core $\lambda$ per element of $b^+$.  

For any signature item or declaration that defines some entity to be equal to $A$ with classification annotation $B$ (e.g., $\mt{val} \; x : B = A$), $B$ and the preceding colon (or similar punctuation) may be omitted, in which case it is filled in as a wildcard.

A signature item or declaration $\mt{type} \; x$ or $\mt{type} \; x = \tau$ is elaborated into $\mt{con} \; x :: \mt{Type}$ or $\mt{con} \; x :: \mt{Type} = \tau$, respectively.

A signature item or declaration $\mt{class} \; x = \lambda y :: \mt{Type} \Rightarrow c$ may be abbreviated $\mt{class} \; x \; y = c$.

Handling of implicit and explicit constructor arguments may be tweaked with some prefixes to variable references.  An expression $@x$ is a version of $x$ where all implicit constructor arguments have been made explicit.  An expression $@@x$ achieves the same effect, additionally halting automatic resolution of type class instances.  The same syntax works for variables projected out of modules and for capitalized variables (datatype constructors).

At the expression level, an analogue is available of the composite $\lambda$ form for constructors.  We define the language of binders as $b ::= x \mid (x : \tau) \mid (x \; ? \; \kappa) \mid [c \sim c]$.  A lone variable $x$ as a binder stands for an expression variable of unspecified type.

A $\mt{val}$ or $\mt{val} \; \mt{rec}$ declaration may include expression binders before the equal sign, following the binder grammar from the last paragraph.  Such declarations are elaborated into versions that add additional $\lambda$s to the fronts of the righthand sides, as appropriate.  The keyword $\mt{fun}$ is a synonym for $\mt{val} \; \mt{rec}$.

A signature item $\mt{functor} \; X_1 \; (X_2 : S_1) : S_2$ is elaborated into $\mt{structure} \; X_1 : \mt{functor}(X_2 : S_1) : S_2$.  A declaration $\mt{functor} \; X_1 \; (X_2 : S_1) : S_2 = M$ is elaborated into $\mt{structure} \; X_1 : \mt{functor}(X_2 : S_1) : S_2 = \mt{functor}(X_2 : S_1) : S_2 = M$.

A declaration $\mt{table} \; x : \{(c = c,)^*\}$ is elaborated into $\mt{table} \; x : [(c = c,)^*]$

The syntax $\mt{where} \; \mt{type}$ is an alternate form of $\mt{where} \; \mt{con}$.

The syntax $\mt{if} \; e \; \mt{then} \; e_1 \; \mt{else} \; e_2$ expands to $\mt{case} \; e \; \mt{of} \; \mt{Basis}.\mt{True} \Rightarrow e_1 \mid \mt{Basis}.\mt{False} \Rightarrow e_2$.

There are infix operator syntaxes for a number of functions defined in the $\mt{Basis}$ module.  There is $=$ for $\mt{eq}$, $\neq$ for $\mt{neq}$, $-$ for $\mt{neg}$ (as a prefix operator) and $\mt{minus}$, $+$ for $\mt{plus}$, $\times$ for $\mt{times}$, $/$ for $\mt{div}$, $\%$ for $\mt{mod}$, $<$ for $\mt{lt}$, $\leq$ for $\mt{le}$, $>$ for $\mt{gt}$, and $\geq$ for $\mt{ge}$.

A signature item $\mt{table} \; x : c$ is shorthand for $\mt{val} \; x : \mt{Basis}.\mt{sql\_table} \; c$.  $\mt{sequence} \; x$ is short for $\mt{val} \; x : \mt{Basis}.\mt{sql\_sequence}$, and $\mt{cookie} \; x : \tau$ is shorthand for $\mt{val} \; x : \mt{Basis}.\mt{http\_cookie} \; \tau$.


\section{Static Semantics}

In this section, we give a declarative presentation of Ur's typing rules and related judgments.  Inference is the subject of the next section; here, we assume that an oracle has filled in all wildcards with concrete values.

Since there is significant mutual recursion among the judgments, we introduce them all before beginning to give rules.  We use the same variety of contexts throughout this section, implicitly introducing new sorts of context entries as needed.
\begin{itemize}
\item $\Gamma \vdash c :: \kappa$ assigns a kind to a constructor in a context.
\item $\Gamma \vdash c \sim c$ proves the disjointness of two record constructors; that is, that they share no field names.  We overload the judgment to apply to pairs of field names as well.
\item $\Gamma \vdash c \hookrightarrow C$ proves that record constructor $c$ decomposes into set $C$ of field names and record constructors.
\item $\Gamma \vdash c \equiv c$ proves the computational equivalence of two constructors.  This is often called a \emph{definitional equality} in the world of type theory.
\item $\Gamma \vdash e : \tau$ is a standard typing judgment.
\item $\Gamma \vdash p \leadsto \Gamma; \tau$ combines typing of patterns with calculation of which new variables they bind.
\item $\Gamma \vdash d \leadsto \Gamma$ expresses how a declaration modifies a context.  We overload this judgment to apply to sequences of declarations, as well as to signature items and sequences of signature items.
\item $\Gamma \vdash S \equiv S$ is the signature equivalence judgment.
\item $\Gamma \vdash S \leq S$ is the signature compatibility judgment.  We write $\Gamma \vdash S$ as shorthand for $\Gamma \vdash S \leq S$.
\item $\Gamma \vdash M : S$ is the module signature checking judgment.
\item $\mt{proj}(M, \overline{s}, V)$ is a partial function for projecting a signature item from $\overline{s}$, given the module $M$ that we project from.  $V$ may be $\mt{con} \; x$, $\mt{datatype} \; x$, $\mt{val} \; x$, $\mt{signature} \; X$, or $\mt{structure} \; X$.  The parameter $M$ is needed because the projected signature item may refer to other items from $\overline{s}$.
\item $\mt{selfify}(M, \overline{s})$ adds information to signature items $\overline{s}$ to reflect the fact that we are concerned with the particular module $M$.  This function is overloaded to work over individual signature items as well.
\end{itemize}

\subsection{Kinding}

$$\infer{\Gamma \vdash (c) :: \kappa :: \kappa}{
  \Gamma \vdash c :: \kappa
}
\quad \infer{\Gamma \vdash x :: \kappa}{
  x :: \kappa \in \Gamma
}
\quad \infer{\Gamma \vdash x :: \kappa}{
  x :: \kappa = c \in \Gamma
}$$

$$\infer{\Gamma \vdash M.x :: \kappa}{
  \Gamma \vdash M : \mt{sig} \; \overline{s} \; \mt{end}
  & \mt{proj}(M, \overline{s}, \mt{con} \; x) = \kappa
}
\quad \infer{\Gamma \vdash M.x :: \kappa}{
  \Gamma \vdash M : \mt{sig} \; \overline{s} \; \mt{end}
  & \mt{proj}(M, \overline{s}, \mt{con} \; x) = (\kappa, c)
}$$

$$\infer{\Gamma \vdash \tau_1 \to \tau_2 :: \mt{Type}}{
  \Gamma \vdash \tau_1 :: \mt{Type}
  & \Gamma \vdash \tau_2 :: \mt{Type}
}
\quad \infer{\Gamma \vdash x \; ? \: \kappa \to \tau :: \mt{Type}}{
  \Gamma, x :: \kappa \vdash \tau :: \mt{Type}
}
\quad \infer{\Gamma \vdash \$c :: \mt{Type}}{
  \Gamma \vdash c :: \{\mt{Type}\}
}$$

$$\infer{\Gamma \vdash c_1 \; c_2 :: \kappa_2}{
  \Gamma \vdash c_1 :: \kappa_1 \to \kappa_2
  & \Gamma \vdash c_2 :: \kappa_1
}
\quad \infer{\Gamma \vdash \lambda x \; :: \; \kappa_1 \Rightarrow c :: \kappa_1 \to \kappa_2}{
  \Gamma, x :: \kappa_1 \vdash c :: \kappa_2
}$$

$$\infer{\Gamma \vdash () :: \mt{Unit}}{}
\quad \infer{\Gamma \vdash \#X :: \mt{Name}}{}$$

$$\infer{\Gamma \vdash [\overline{c_i = c'_i}] :: \{\kappa\}}{
  \forall i: \Gamma \vdash c_i : \mt{Name}
  & \Gamma \vdash c'_i :: \kappa
  & \forall i \neq j: \Gamma \vdash c_i \sim c_j
}
\quad \infer{\Gamma \vdash c_1 \rc c_2 :: \{\kappa\}}{
  \Gamma \vdash c_1 :: \{\kappa\}
  & \Gamma \vdash c_2 :: \{\kappa\}
  & \Gamma \vdash c_1 \sim c_2
}$$

$$\infer{\Gamma \vdash \mt{fold} :: ((\mt{Name} \to \kappa_1 \to \kappa_2 \to \kappa_2) \to \kappa_2 \to \{\kappa_1\} \to \kappa_2}{}$$

$$\infer{\Gamma \vdash (\overline c) :: (k_1 \times \ldots \times k_n)}{
  \forall i: \Gamma \vdash c_i :: k_i
}
\quad \infer{\Gamma \vdash c.i :: k_i}{
  \Gamma \vdash c :: (k_1 \times \ldots \times k_n)
}$$

$$\infer{\Gamma \vdash \lambda [c_1 \sim c_2] \Rightarrow c :: \kappa}{
  \Gamma \vdash c_1 :: \{\kappa'\}
  & \Gamma \vdash c_2 :: \{\kappa'\}
  & \Gamma, c_1 \sim c_2 \vdash c :: \kappa
}$$

\subsection{Record Disjointness}

We will use a keyword $\mt{map}$ as a shorthand, such that, for $f$ of kind $\kappa \to \kappa'$, $\mt{map} \; f$ stands for $\mt{fold} \; (\lambda (x_1 :: \mt{Name}) (x_2 :: \kappa) (x_3 :: \{\kappa'\}) \Rightarrow [x_1 = f \; x_2] \rc x_3) \; []$.

$$\infer{\Gamma \vdash c_1 \sim c_2}{
  \Gamma \vdash c_1 \hookrightarrow C_1
  & \Gamma \vdash c_2 \hookrightarrow C_2
  & \forall c'_1 \in C_1, c'_2 \in C_2: \Gamma \vdash c'_1 \sim c'_2
}
\quad \infer{\Gamma \vdash X \sim X'}{
  X \neq X'
}$$

$$\infer{\Gamma \vdash c_1 \sim c_2}{
  c'_1 \sim c'_2 \in \Gamma
  & \Gamma \vdash c'_1 \hookrightarrow C_1
  & \Gamma \vdash c'_2 \hookrightarrow C_2
  & c_1 \in C_1
  & c_2 \in C_2
}$$

$$\infer{\Gamma \vdash c \hookrightarrow \{c\}}{}
\quad \infer{\Gamma \vdash [\overline{c = c'}] \hookrightarrow \{\overline{c}\}}{}
\quad \infer{\Gamma \vdash c_1 \rc c_2 \hookrightarrow C_1 \cup C_2}{
  \Gamma \vdash c_1 \hookrightarrow C_1
  & \Gamma \vdash c_2 \hookrightarrow C_2
}
\quad \infer{\Gamma \vdash c \hookrightarrow C}{
  \Gamma \vdash c \equiv c'
  & \Gamma \vdash c' \hookrightarrow C
}
\quad \infer{\Gamma \vdash \mt{map} \; f \; c \hookrightarrow C}{
  \Gamma \vdash c \hookrightarrow C
}$$

\subsection{\label{definitional}Definitional Equality}

We use $\mathcal C$ to stand for a one-hole context that, when filled, yields a constructor.  The notation $\mathcal C[c]$ plugs $c$ into $\mathcal C$.  We omit the standard definition of one-hole contexts.  We write $[x \mapsto c_1]c_2$ for capture-avoiding substitution of $c_1$ for $x$ in $c_2$.

$$\infer{\Gamma \vdash c \equiv c}{}
\quad \infer{\Gamma \vdash c_1 \equiv c_2}{
  \Gamma \vdash c_2 \equiv c_1
}
\quad \infer{\Gamma \vdash c_1 \equiv c_3}{
  \Gamma \vdash c_1 \equiv c_2
  & \Gamma \vdash c_2 \equiv c_3
}
\quad \infer{\Gamma \vdash \mathcal C[c_1] \equiv \mathcal C[c_2]}{
  \Gamma \vdash c_1 \equiv c_2
}$$

$$\infer{\Gamma \vdash x \equiv c}{
  x :: \kappa = c \in \Gamma
}
\quad \infer{\Gamma \vdash M.x \equiv c}{
  \Gamma \vdash M : \mt{sig} \; \overline{s} \; \mt{end}
  & \mt{proj}(M, \overline{s}, \mt{con} \; x) = (\kappa, c)
}
\quad \infer{\Gamma \vdash (\overline c).i \equiv c_i}{}$$

$$\infer{\Gamma \vdash (\lambda x :: \kappa \Rightarrow c) \; c' \equiv [x \mapsto c'] c}{}
\quad \infer{\Gamma \vdash c_1 \rc c_2 \equiv c_2 \rc c_1}{}
\quad \infer{\Gamma \vdash c_1 \rc (c_2 \rc c_3) \equiv (c_1 \rc c_2) \rc c_3}{}$$

$$\infer{\Gamma \vdash [] \rc c \equiv c}{}
\quad \infer{\Gamma \vdash [\overline{c_1 = c'_1}] \rc [\overline{c_2 = c'_2}] \equiv [\overline{c_1 = c'_1}, \overline{c_2 = c'_2}]}{}$$

$$\infer{\Gamma \vdash \lambda [c_1 \sim c_2] \Rightarrow c \equiv c}{
  \Gamma \vdash c_1 \sim c_2
}
\quad \infer{\Gamma \vdash \mt{fold} \; f \; i \; [] \equiv i}{}
\quad \infer{\Gamma \vdash \mt{fold} \; f \; i \; ([c_1 = c_2] \rc c) \equiv f \; c_1 \; c_2 \; (\mt{fold} \; f \; i \; c)}{}$$

$$\infer{\Gamma \vdash \mt{map} \; (\lambda x \Rightarrow x) \; c \equiv c}{}
\quad \infer{\Gamma \vdash \mt{fold} \; f \; i \; (\mt{map} \; f' \; c)
  \equiv \mt{fold} \; (\lambda (x_1 :: \mt{Name}) (x_2 :: \kappa) \Rightarrow f \; x_1 \; (f' \; x_2)) \; i \; c}{}$$

$$\infer{\Gamma \vdash \mt{map} \; f \; (c_1 \rc c_2) \equiv \mt{map} \; f \; c_1 \rc \mt{map} \; f \; c_2}{}$$

\subsection{Expression Typing}

We assume the existence of a function $T$ assigning types to literal constants.  It maps integer constants to $\mt{Basis}.\mt{int}$, float constants to $\mt{Basis}.\mt{float}$, and string constants to $\mt{Basis}.\mt{string}$.

We also refer to a function $\mathcal I$, such that $\mathcal I(\tau)$ ``uses an oracle'' to instantiate all constructor function arguments at the beginning of $\tau$ that are marked implicit; i.e., replace $x_1 ::: \kappa_1 \to \ldots \to x_n ::: \kappa_n \to \tau$ with $[x_1 \mapsto c_1]\ldots[x_n \mapsto c_n]\tau$, where the $c_i$s are inferred and $\tau$ does not start like $x ::: \kappa \to \tau'$.

$$\infer{\Gamma \vdash e : \tau : \tau}{
  \Gamma \vdash e : \tau
}
\quad \infer{\Gamma \vdash e : \tau}{
  \Gamma \vdash e : \tau'
  & \Gamma \vdash \tau' \equiv \tau
}
\quad \infer{\Gamma \vdash \ell : T(\ell)}{}$$

$$\infer{\Gamma \vdash x : \mathcal I(\tau)}{
  x : \tau \in \Gamma
}
\quad \infer{\Gamma \vdash M.x : \mathcal I(\tau)}{
  \Gamma \vdash M : \mt{sig} \; \overline{s} \; \mt{end}
  & \mt{proj}(M, \overline{s}, \mt{val} \; x) = \tau
}
\quad \infer{\Gamma \vdash X : \mathcal I(\tau)}{
  X : \tau \in \Gamma
}
\quad \infer{\Gamma \vdash M.X : \mathcal I(\tau)}{
  \Gamma \vdash M : \mt{sig} \; \overline{s} \; \mt{end}
  & \mt{proj}(M, \overline{s}, \mt{val} \; X) = \tau
}$$

$$\infer{\Gamma \vdash e_1 \; e_2 : \tau_2}{
  \Gamma \vdash e_1 : \tau_1 \to \tau_2
  & \Gamma \vdash e_2 : \tau_1
}
\quad \infer{\Gamma \vdash \lambda x : \tau_1 \Rightarrow e : \tau_1 \to \tau_2}{
  \Gamma, x : \tau_1 \vdash e : \tau_2
}$$

$$\infer{\Gamma \vdash e [c] : [x \mapsto c]\tau}{
  \Gamma \vdash e : x :: \kappa \to \tau
  & \Gamma \vdash c :: \kappa
}
\quad \infer{\Gamma \vdash \lambda x \; ? \; \kappa \Rightarrow e : x \; ? \; \kappa \to \tau}{
  \Gamma, x :: \kappa \vdash e : \tau
}$$

$$\infer{\Gamma \vdash \{\overline{c = e}\} : \{\overline{c : \tau}\}}{
  \forall i: \Gamma \vdash c_i :: \mt{Name}
  & \Gamma \vdash e_i : \tau_i
  & \forall i \neq j: \Gamma \vdash c_i \sim c_j
}
\quad \infer{\Gamma \vdash e.c : \tau}{
  \Gamma \vdash e : \$([c = \tau] \rc c')
}
\quad \infer{\Gamma \vdash e_1 \rc e_2 : \$(c_1 \rc c_2)}{
  \Gamma \vdash e_1 : \$c_1
  & \Gamma \vdash e_2 : \$c_2
}$$

$$\infer{\Gamma \vdash e \rcut c : \$c'}{
  \Gamma \vdash e : \$([c = \tau] \rc c')
}
\quad \infer{\Gamma \vdash e \rcutM c : \$c'}{
  \Gamma \vdash e : \$(c \rc c')
}$$

$$\infer{\Gamma \vdash \mt{fold} : \begin{array}{c}
    x_1 :: (\{\kappa\} \to \tau)
    \to (x_2 :: \mt{Name} \to x_3 :: \kappa \to x_4 :: \{\kappa\} \to \lambda [[x_2] \sim x_4]
    \Rightarrow x_1 \; x_4 \to x_1 \; ([x_2 = x_3] \rc x_4)) \\
    \to x_1 \; [] \to x_5 :: \{\kappa\} \to x_1 \; x_5
  \end{array}}{}$$

$$\infer{\Gamma \vdash \mt{let} \; \overline{ed} \; \mt{in} \; e \; \mt{end} : \tau}{
  \Gamma \vdash \overline{ed} \leadsto \Gamma'
  & \Gamma' \vdash e : \tau
}
\quad \infer{\Gamma \vdash \mt{case} \; e \; \mt{of} \; \overline{p \Rightarrow e} : \tau}{
  \forall i: \Gamma \vdash p_i \leadsto \Gamma_i, \tau'
  & \Gamma_i \vdash e_i : \tau
}$$

$$\infer{\Gamma \vdash [c_1 \sim c_2] \Rightarrow e : [c_1 \sim c_2] \Rightarrow \tau}{
  \Gamma \vdash c_1 :: \{\kappa\}
  & \Gamma \vdash c_2 :: \{\kappa\}
  & \Gamma, c_1 \sim c_2 \vdash e : \tau
}$$

\subsection{Pattern Typing}

$$\infer{\Gamma \vdash \_ \leadsto \Gamma; \tau}{}
\quad \infer{\Gamma \vdash x \leadsto \Gamma, x : \tau; \tau}{}
\quad \infer{\Gamma \vdash \ell \leadsto \Gamma; T(\ell)}{}$$

$$\infer{\Gamma \vdash X \leadsto \Gamma; \overline{[x_i \mapsto \tau'_i]}\tau}{
  X : \overline{x ::: \mt{Type}} \to \tau \in \Gamma
  & \textrm{$\tau$ not a function type}
}
\quad \infer{\Gamma \vdash X \; p \leadsto \Gamma'; \overline{[x_i \mapsto \tau'_i]}\tau}{
  X : \overline{x ::: \mt{Type}} \to \tau'' \to \tau \in \Gamma
  & \Gamma \vdash p \leadsto \Gamma'; \overline{[x_i \mapsto \tau'_i]}\tau''
}$$

$$\infer{\Gamma \vdash M.X \leadsto \Gamma; \overline{[x_i \mapsto \tau'_i]}\tau}{
  \Gamma \vdash M : \mt{sig} \; \overline{s} \; \mt{end}
  & \mt{proj}(M, \overline{s}, \mt{val} \; X) = \overline{x ::: \mt{Type}} \to \tau
  & \textrm{$\tau$ not a function type}
}$$

$$\infer{\Gamma \vdash M.X \; p \leadsto \Gamma'; \overline{[x_i \mapsto \tau'_i]}\tau}{
  \Gamma \vdash M : \mt{sig} \; \overline{s} \; \mt{end}
  & \mt{proj}(M, \overline{s}, \mt{val} \; X) = \overline{x ::: \mt{Type}} \to \tau'' \to \tau
  & \Gamma \vdash p \leadsto \Gamma'; \overline{[x_i \mapsto \tau'_i]}\tau''
}$$

$$\infer{\Gamma \vdash \{\overline{x = p}\} \leadsto \Gamma_n; \{\overline{x = \tau}\}}{
  \Gamma_0 = \Gamma
  & \forall i: \Gamma_i \vdash p_i \leadsto \Gamma_{i+1}; \tau_i
}
\quad \infer{\Gamma \vdash \{\overline{x = p}, \ldots\} \leadsto \Gamma_n; \$([\overline{x = \tau}] \rc c)}{
  \Gamma_0 = \Gamma
  & \forall i: \Gamma_i \vdash p_i \leadsto \Gamma_{i+1}; \tau_i
}$$

\subsection{Declaration Typing}

We use an auxiliary judgment $\overline{y}; x; \Gamma \vdash \overline{dc} \leadsto \Gamma'$, expressing the enrichment of $\Gamma$ with the types of the datatype constructors $\overline{dc}$, when they are known to belong to datatype $x$ with type parameters $\overline{y}$.

This is the first judgment where we deal with type classes, for the $\mt{class}$ declaration form.  We will omit their special handling in this formal specification.  Section \ref{typeclasses} gives an informal description of how type classes influence type inference.

We presuppose the existence of a function $\mathcal O$, where $\mathcal O(M, \overline{s})$ implements the $\mt{open}$ declaration by producing a context with the appropriate entry for each available component of module $M$ with signature items $\overline{s}$.  Where possible, $\mathcal O$ uses ``transparent'' entries (e.g., an abstract type $M.x$ is mapped to $x :: \mt{Type} = M.x$), so that the relationship with $M$ is maintained.  A related function $\mathcal O_c$ builds a context containing the disjointness constraints found in $\overline s$.
We write $\kappa_1^n \to \kappa$ as a shorthand, where $\kappa_1^0 \to \kappa = \kappa$ and $\kappa_1^{n+1} \to \kappa_2 = \kappa_1 \to (\kappa_1^n \to \kappa_2)$.  We write $\mt{len}(\overline{y})$ for the length of vector $\overline{y}$ of variables.

$$\infer{\Gamma \vdash \cdot \leadsto \Gamma}{}
\quad \infer{\Gamma \vdash d, \overline{d} \leadsto \Gamma''}{
  \Gamma \vdash d \leadsto \Gamma'
  & \Gamma' \vdash \overline{d} \leadsto \Gamma''
}$$

$$\infer{\Gamma \vdash \mt{con} \; x :: \kappa = c \leadsto \Gamma, x :: \kappa = c}{
  \Gamma \vdash c :: \kappa
}
\quad \infer{\Gamma \vdash \mt{datatype} \; x \; \overline{y} = \overline{dc} \leadsto \Gamma'}{
  \overline{y}; x; \Gamma, x :: \mt{Type}^{\mt{len}(\overline y)} \to \mt{Type} \vdash \overline{dc} \leadsto \Gamma'
}$$

$$\infer{\Gamma \vdash \mt{datatype} \; x = \mt{datatype} \; M.z \leadsto \Gamma'}{
  \Gamma \vdash M : \mt{sig} \; \overline{s} \; \mt{end}
  & \mt{proj}(M, \overline{s}, \mt{datatype} \; z) = (\overline{y}, \overline{dc})
  & \overline{y}; x; \Gamma, x :: \mt{Type}^{\mt{len}(\overline y)} \to \mt{Type} = M.z \vdash \overline{dc} \leadsto \Gamma'
}$$

$$\infer{\Gamma \vdash \mt{val} \; x : \tau = e \leadsto \Gamma, x : \tau}{
  \Gamma \vdash e : \tau
}$$

$$\infer{\Gamma \vdash \mt{val} \; \mt{rec} \; \overline{x : \tau = e} \leadsto \Gamma, \overline{x : \tau}}{
  \forall i: \Gamma, \overline{x : \tau} \vdash e_i : \tau_i
  & \textrm{$e_i$ starts with an expression $\lambda$, optionally preceded by constructor and disjointness $\lambda$s}
}$$

$$\infer{\Gamma \vdash \mt{structure} \; X : S = M \leadsto \Gamma, X : S}{
  \Gamma \vdash M : S
  & \textrm{ $M$ not a constant or application}
}
\quad \infer{\Gamma \vdash \mt{structure} \; X : S = M \leadsto \Gamma, X : \mt{selfify}(X, \overline{s})}{
  \Gamma \vdash M : \mt{sig} \; \overline{s} \; \mt{end}
}$$

$$\infer{\Gamma \vdash \mt{signature} \; X = S \leadsto \Gamma, X = S}{
  \Gamma \vdash S
}$$

$$\infer{\Gamma \vdash \mt{open} \; M \leadsto \Gamma, \mathcal O(M, \overline{s})}{
  \Gamma \vdash M : \mt{sig} \; \overline{s} \; \mt{end}
}$$

$$\infer{\Gamma \vdash \mt{constraint} \; c_1 \sim c_2 \leadsto \Gamma}{
  \Gamma \vdash c_1 :: \{\kappa\}
  & \Gamma \vdash c_2 :: \{\kappa\}
  & \Gamma \vdash c_1 \sim c_2
}
\quad \infer{\Gamma \vdash \mt{open} \; \mt{constraints} \; M \leadsto \Gamma, \mathcal O_c(M, \overline{s})}{
  \Gamma \vdash M : \mt{sig} \; \overline{s} \; \mt{end}
}$$

$$\infer{\Gamma \vdash \mt{table} \; x : c \leadsto \Gamma, x : \mt{Basis}.\mt{sql\_table} \; c}{
  \Gamma \vdash c :: \{\mt{Type}\}
}
\quad \infer{\Gamma \vdash \mt{sequence} \; x \leadsto \Gamma, x : \mt{Basis}.\mt{sql\_sequence}}{}$$

$$\infer{\Gamma \vdash \mt{cookie} \; x : \tau \leadsto \Gamma, x : \mt{Basis}.\mt{http\_cookie} \; \tau}{
  \Gamma \vdash \tau :: \mt{Type}
}$$

$$\infer{\Gamma \vdash \mt{class} \; x = c \leadsto \Gamma, x :: \mt{Type} \to \mt{Type} = c}{
  \Gamma \vdash c :: \mt{Type} \to \mt{Type}
}$$

$$\infer{\overline{y}; x; \Gamma \vdash \cdot \leadsto \Gamma}{}
\quad \infer{\overline{y}; x; \Gamma \vdash X \mid \overline{dc} \leadsto \Gamma', X : \overline{y ::: \mt{Type}} \to x \; \overline{y}}{
  \overline{y}; x; \Gamma \vdash \overline{dc} \leadsto \Gamma'
}
\quad \infer{\overline{y}; x; \Gamma \vdash X \; \mt{of} \; \tau \mid \overline{dc} \leadsto \Gamma', X : \overline{y ::: \mt{Type}} \to \tau \to x \; \overline{y}}{
  \overline{y}; x; \Gamma \vdash \overline{dc} \leadsto \Gamma'
}$$

\subsection{Signature Item Typing}

We appeal to a signature item analogue of the $\mathcal O$ function from the last subsection.

$$\infer{\Gamma \vdash \cdot \leadsto \Gamma}{}
\quad \infer{\Gamma \vdash s, \overline{s} \leadsto \Gamma''}{
  \Gamma \vdash s \leadsto \Gamma'
  & \Gamma' \vdash \overline{s} \leadsto \Gamma''
}$$

$$\infer{\Gamma \vdash \mt{con} \; x :: \kappa \leadsto \Gamma, x :: \kappa}{}
\quad \infer{\Gamma \vdash \mt{con} \; x :: \kappa = c \leadsto \Gamma, x :: \kappa = c}{
  \Gamma \vdash c :: \kappa
}
\quad \infer{\Gamma \vdash \mt{datatype} \; x \; \overline{y} = \overline{dc} \leadsto \Gamma'}{
  \overline{y}; x; \Gamma, x :: \mt{Type}^{\mt{len}(\overline y)} \to \mt{Type} \vdash \overline{dc} \leadsto \Gamma'
}$$

$$\infer{\Gamma \vdash \mt{datatype} \; x = \mt{datatype} \; M.z \leadsto \Gamma'}{
  \Gamma \vdash M : \mt{sig} \; \overline{s} \; \mt{end}
  & \mt{proj}(M, \overline{s}, \mt{datatype} \; z) = (\overline{y}, \overline{dc})
  & \overline{y}; x; \Gamma, x :: \mt{Type}^{\mt{len}(\overline y)} \to \mt{Type} = M.z \vdash \overline{dc} \leadsto \Gamma'
}$$

$$\infer{\Gamma \vdash \mt{val} \; x : \tau \leadsto \Gamma, x : \tau}{
  \Gamma \vdash \tau :: \mt{Type}
}$$

$$\infer{\Gamma \vdash \mt{structure} \; X : S \leadsto \Gamma, X : S}{
  \Gamma \vdash S
}
\quad \infer{\Gamma \vdash \mt{signature} \; X = S \leadsto \Gamma, X = S}{
  \Gamma \vdash S
}$$

$$\infer{\Gamma \vdash \mt{include} \; S \leadsto \Gamma, \mathcal O(\overline{s})}{
  \Gamma \vdash S
  & \Gamma \vdash S \equiv \mt{sig} \; \overline{s} \; \mt{end}
}$$

$$\infer{\Gamma \vdash \mt{constraint} \; c_1 \sim c_2 \leadsto \Gamma, c_1 \sim c_2}{
  \Gamma \vdash c_1 :: \{\kappa\}
  & \Gamma \vdash c_2 :: \{\kappa\}
}$$

$$\infer{\Gamma \vdash \mt{class} \; x = c \leadsto \Gamma, x :: \mt{Type} \to \mt{Type} = c}{
  \Gamma \vdash c :: \mt{Type} \to \mt{Type}
}
\quad \infer{\Gamma \vdash \mt{class} \; x \leadsto \Gamma, x :: \mt{Type} \to \mt{Type}}{}$$

\subsection{Signature Compatibility}

To simplify the judgments in this section, we assume that all signatures are alpha-varied as necessary to avoid including multiple bindings for the same identifier.  This is in addition to the usual alpha-variation of locally-bound variables.

We rely on a judgment $\Gamma \vdash \overline{s} \leq s'$, which expresses the occurrence in signature items $\overline{s}$ of an item compatible with $s'$.  We also use a judgment $\Gamma \vdash \overline{dc} \leq \overline{dc}$, which expresses compatibility of datatype definitions.

$$\infer{\Gamma \vdash S \equiv S}{}
\quad \infer{\Gamma \vdash S_1 \equiv S_2}{
  \Gamma \vdash S_2 \equiv S_1
}
\quad \infer{\Gamma \vdash X \equiv S}{
  X = S \in \Gamma
}
\quad \infer{\Gamma \vdash M.X \equiv S}{
  \Gamma \vdash M : \mt{sig} \; \overline{s} \; \mt{end}
  & \mt{proj}(M, \overline{s}, \mt{signature} \; X) = S
}$$

$$\infer{\Gamma \vdash S \; \mt{where} \; \mt{con} \; x = c \equiv \mt{sig} \; \overline{s^1} \; \mt{con} \; x :: \kappa = c \; \overline{s_2} \; \mt{end}}{
  \Gamma \vdash S \equiv \mt{sig} \; \overline{s^1} \; \mt{con} \; x :: \kappa \; \overline{s_2} \; \mt{end}
  & \Gamma \vdash c :: \kappa
}
\quad \infer{\Gamma \vdash \mt{sig} \; \overline{s^1} \; \mt{include} \; S \; \overline{s^2} \; \mt{end} \equiv \mt{sig} \; \overline{s^1} \; \overline{s} \; \overline{s^2} \; \mt{end}}{
  \Gamma \vdash S \equiv \mt{sig} \; \overline{s} \; \mt{end}
}$$

$$\infer{\Gamma \vdash S_1 \leq S_2}{
  \Gamma \vdash S_1 \equiv S_2
}
\quad \infer{\Gamma \vdash \mt{sig} \; \overline{s} \; \mt{end} \leq \mt{sig} \; \mt{end}}{}
\quad \infer{\Gamma \vdash \mt{sig} \; \overline{s} \; \mt{end} \leq \mt{sig} \; s' \; \overline{s'} \; \mt{end}}{
  \Gamma \vdash \overline{s} \leq s'
  & \Gamma \vdash s' \leadsto \Gamma'
  & \Gamma' \vdash \mt{sig} \; \overline{s} \; \mt{end} \leq \mt{sig} \; \overline{s'} \; \mt{end}
}$$

$$\infer{\Gamma \vdash s \; \overline{s} \leq s'}{
  \Gamma \vdash s \leq s'
}
\quad \infer{\Gamma \vdash s \; \overline{s} \leq s'}{
  \Gamma \vdash s \leadsto \Gamma'
  & \Gamma' \vdash \overline{s} \leq s'
}$$

$$\infer{\Gamma \vdash \mt{functor} (X : S_1) : S_2 \leq \mt{functor} (X : S'_1) : S'_2}{
  \Gamma \vdash S'_1 \leq S_1
  & \Gamma, X : S'_1 \vdash S_2 \leq S'_2
}$$

$$\infer{\Gamma \vdash \mt{con} \; x :: \kappa \leq \mt{con} \; x :: \kappa}{}
\quad \infer{\Gamma \vdash \mt{con} \; x :: \kappa = c \leq \mt{con} \; x :: \kappa}{}
\quad \infer{\Gamma \vdash \mt{datatype} \; x \; \overline{y} = \overline{dc} \leq \mt{con} \; x :: \mt{Type}^{\mt{len}(\overline y)} \to \mt{Type}}{}$$

$$\infer{\Gamma \vdash \mt{datatype} \; x = \mt{datatype} \; M.z \leq \mt{con} \; x :: \mt{Type}^{\mt{len}(y)} \to \mt{Type}}{
  \Gamma \vdash M : \mt{sig} \; \overline{s} \; \mt{end}
  & \mt{proj}(M, \overline{s}, \mt{datatype} \; z) = (\overline{y}, \overline{dc})
}$$

$$\infer{\Gamma \vdash \mt{class} \; x \leq \mt{con} \; x :: \mt{Type} \to \mt{Type}}{}
\quad \infer{\Gamma \vdash \mt{class} \; x = c \leq \mt{con} \; x :: \mt{Type} \to \mt{Type}}{}$$

$$\infer{\Gamma \vdash \mt{con} \; x :: \kappa = c_1 \leq \mt{con} \; x :: \mt{\kappa} = c_2}{
  \Gamma \vdash c_1 \equiv c_2
}
\quad \infer{\Gamma \vdash \mt{class} \; x = c_1 \leq \mt{con} \; x :: \mt{Type} \to \mt{Type} = c_2}{
  \Gamma \vdash c_1 \equiv c_2
}$$

$$\infer{\Gamma \vdash \mt{datatype} \; x \; \overline{y} = \overline{dc} \leq \mt{datatype} \; x \; \overline{y} = \overline{dc'}}{
  \Gamma, \overline{y :: \mt{Type}} \vdash \overline{dc} \leq \overline{dc'}
}$$

$$\infer{\Gamma \vdash \mt{datatype} \; x = \mt{datatype} \; M.z \leq \mt{datatype} \; x \; \overline{y} = \overline{dc'}}{
  \Gamma \vdash M : \mt{sig} \; \overline{s} \; \mt{end}
  & \mt{proj}(M, \overline{s}, \mt{datatype} \; z) = (\overline{y}, \overline{dc})
  & \Gamma, \overline{y :: \mt{Type}} \vdash \overline{dc} \leq \overline{dc'}
}$$

$$\infer{\Gamma \vdash \cdot \leq \cdot}{}
\quad \infer{\Gamma \vdash X; \overline{dc} \leq X; \overline{dc'}}{
  \Gamma \vdash \overline{dc} \leq \overline{dc'}
}
\quad \infer{\Gamma \vdash X \; \mt{of} \; \tau_1; \overline{dc} \leq X \; \mt{of} \; \tau_2; \overline{dc'}}{
  \Gamma \vdash \tau_1 \equiv \tau_2
  & \Gamma \vdash \overline{dc} \leq \overline{dc'}
}$$

$$\infer{\Gamma \vdash \mt{datatype} \; x = \mt{datatype} \; M.z \leq \mt{datatype} \; x = \mt{datatype} \; M'.z'}{
  \Gamma \vdash M.z \equiv M'.z'
}$$

$$\infer{\Gamma \vdash \mt{val} \; x : \tau_1 \leq \mt{val} \; x : \tau_2}{
  \Gamma \vdash \tau_1 \equiv \tau_2
}
\quad \infer{\Gamma \vdash \mt{structure} \; X : S_1 \leq \mt{structure} \; X : S_2}{
  \Gamma \vdash S_1 \leq S_2
}
\quad \infer{\Gamma \vdash \mt{signature} \; X = S_1 \leq \mt{signature} \; X = S_2}{
  \Gamma \vdash S_1 \leq S_2
  & \Gamma \vdash S_2 \leq S_1
}$$

$$\infer{\Gamma \vdash \mt{constraint} \; c_1 \sim c_2 \leq \mt{constraint} \; c'_1 \sim c'_2}{
  \Gamma \vdash c_1 \equiv c'_1
  & \Gamma \vdash c_2 \equiv c'_2
}$$

$$\infer{\Gamma \vdash \mt{class} \; x \leq \mt{class} \; x}{}
\quad \infer{\Gamma \vdash \mt{class} \; x = c \leq \mt{class} \; x}{}
\quad \infer{\Gamma \vdash \mt{class} \; x = c_1 \leq \mt{class} \; x = c_2}{
  \Gamma \vdash c_1 \equiv c_2
}$$

\subsection{Module Typing}

We use a helper function $\mt{sigOf}$, which converts declarations and sequences of declarations into their principal signature items and sequences of signature items, respectively.

$$\infer{\Gamma \vdash M : S}{
  \Gamma \vdash M : S'
  & \Gamma \vdash S' \leq S
}
\quad \infer{\Gamma \vdash \mt{struct} \; \overline{d} \; \mt{end} : \mt{sig} \; \mt{sigOf}(\overline{d}) \; \mt{end}}{
  \Gamma \vdash \overline{d} \leadsto \Gamma'
}
\quad \infer{\Gamma \vdash X : S}{
  X : S \in \Gamma
}$$

$$\infer{\Gamma \vdash M.X : S}{
  \Gamma \vdash M : \mt{sig} \; \overline{s} \; \mt{end}
  & \mt{proj}(M, \overline{s}, \mt{structure} \; X) = S
}$$

$$\infer{\Gamma \vdash M_1(M_2) : [X \mapsto M_2]S_2}{
  \Gamma \vdash M_1 : \mt{functor}(X : S_1) : S_2
  & \Gamma \vdash M_2 : S_1
}
\quad \infer{\Gamma \vdash \mt{functor} (X : S_1) : S_2 = M : \mt{functor} (X : S_1) : S_2}{
  \Gamma \vdash S_1
  & \Gamma, X : S_1 \vdash S_2
  & \Gamma, X : S_1 \vdash M : S_2
}$$

\begin{eqnarray*}
  \mt{sigOf}(\cdot) &=& \cdot \\
  \mt{sigOf}(s \; \overline{s'}) &=& \mt{sigOf}(s) \; \mt{sigOf}(\overline{s'}) \\
  \\
  \mt{sigOf}(\mt{con} \; x :: \kappa = c) &=& \mt{con} \; x :: \kappa = c \\
  \mt{sigOf}(\mt{datatype} \; x \; \overline{y} = \overline{dc}) &=& \mt{datatype} \; x \; \overline{y} = \overline{dc} \\
  \mt{sigOf}(\mt{datatype} \; x = \mt{datatype} \; M.z) &=& \mt{datatype} \; x = \mt{datatype} \; M.z \\
  \mt{sigOf}(\mt{val} \; x : \tau = e) &=& \mt{val} \; x : \tau \\
  \mt{sigOf}(\mt{val} \; \mt{rec} \; \overline{x : \tau = e}) &=& \overline{\mt{val} \; x : \tau} \\
  \mt{sigOf}(\mt{structure} \; X : S = M) &=& \mt{structure} \; X : S \\
  \mt{sigOf}(\mt{signature} \; X = S) &=& \mt{signature} \; X = S \\
  \mt{sigOf}(\mt{open} \; M) &=& \mt{include} \; S \textrm{ (where $\Gamma \vdash M : S$)} \\
  \mt{sigOf}(\mt{constraint} \; c_1 \sim c_2) &=& \mt{constraint} \; c_1 \sim c_2 \\
  \mt{sigOf}(\mt{open} \; \mt{constraints} \; M) &=& \cdot \\
  \mt{sigOf}(\mt{table} \; x : c) &=& \mt{table} \; x : c \\
  \mt{sigOf}(\mt{sequence} \; x) &=& \mt{sequence} \; x \\
  \mt{sigOf}(\mt{cookie} \; x : \tau) &=& \mt{cookie} \; x : \tau \\
  \mt{sigOf}(\mt{class} \; x = c) &=& \mt{class} \; x = c \\
\end{eqnarray*}
\begin{eqnarray*}
  \mt{selfify}(M, \cdot) &=& \cdot \\
  \mt{selfify}(M, s \; \overline{s'}) &=& \mt{selfify}(M, s) \; \mt{selfify}(M, \overline{s'}) \\
  \\
  \mt{selfify}(M, \mt{con} \; x :: \kappa) &=& \mt{con} \; x :: \kappa = M.x \\
  \mt{selfify}(M, \mt{con} \; x :: \kappa = c) &=& \mt{con} \; x :: \kappa = c \\
  \mt{selfify}(M, \mt{datatype} \; x \; \overline{y} = \overline{dc}) &=& \mt{datatype} \; x \; \overline{y} = \mt{datatype} \; M.x \\
  \mt{selfify}(M, \mt{datatype} \; x = \mt{datatype} \; M'.z) &=& \mt{datatype} \; x = \mt{datatype} \; M'.z \\
  \mt{selfify}(M, \mt{val} \; x : \tau) &=& \mt{val} \; x : \tau \\
  \mt{selfify}(M, \mt{structure} \; X : S) &=& \mt{structure} \; X : \mt{selfify}(M.X, \overline{s}) \textrm{ (where $\Gamma \vdash S \equiv \mt{sig} \; \overline{s} \; \mt{end}$)} \\
  \mt{selfify}(M, \mt{signature} \; X = S) &=& \mt{signature} \; X = S \\
  \mt{selfify}(M, \mt{include} \; S) &=& \mt{include} \; S \\
  \mt{selfify}(M, \mt{constraint} \; c_1 \sim c_2) &=& \mt{constraint} \; c_1 \sim c_2 \\
  \mt{selfify}(M, \mt{class} \; x) &=& \mt{class} \; x = M.x \\
  \mt{selfify}(M, \mt{class} \; x = c) &=& \mt{class} \; x = c \\
\end{eqnarray*}

\subsection{Module Projection}

\begin{eqnarray*}
  \mt{proj}(M, \mt{con} \; x :: \kappa \; \overline{s}, \mt{con} \; x) &=& \kappa \\
  \mt{proj}(M, \mt{con} \; x :: \kappa = c \; \overline{s}, \mt{con} \; x) &=& (\kappa, c) \\
  \mt{proj}(M, \mt{datatype} \; x \; \overline{y} = \overline{dc} \; \overline{s}, \mt{con} \; x) &=& \mt{Type}^{\mt{len}(\overline{y})} \to \mt{Type} \\
  \mt{proj}(M, \mt{datatype} \; x = \mt{datatype} \; M'.z \; \overline{s}, \mt{con} \; x) &=& (\mt{Type}^{\mt{len}(\overline{y})} \to \mt{Type}, M'.z) \textrm{ (where $\Gamma \vdash M' : \mt{sig} \; \overline{s'} \; \mt{end}$} \\
  && \textrm{and $\mt{proj}(M', \overline{s'}, \mt{datatype} \; z) = (\overline{y}, \overline{dc})$)} \\
  \mt{proj}(M, \mt{class} \; x \; \overline{s}, \mt{con} \; x) &=& \mt{Type} \to \mt{Type} \\
  \mt{proj}(M, \mt{class} \; x = c \; \overline{s}, \mt{con} \; x) &=& (\mt{Type} \to \mt{Type}, c) \\
  \\
  \mt{proj}(M, \mt{datatype} \; x \; \overline{y} = \overline{dc} \; \overline{s}, \mt{datatype} \; x) &=& (\overline{y}, \overline{dc}) \\
  \mt{proj}(M, \mt{datatype} \; x = \mt{datatype} \; M'.z \; \overline{s}, \mt{con} \; x) &=& \mt{proj}(M', \overline{s'}, \mt{datatype} \; z) \textrm{ (where $\Gamma \vdash M' : \mt{sig} \; \overline{s'} \; \mt{end}$)} \\
  \\
  \mt{proj}(M, \mt{val} \; x : \tau \; \overline{s}, \mt{val} \; x) &=& \tau \\
  \mt{proj}(M, \mt{datatype} \; x \; \overline{y} = \overline{dc} \; \overline{s}, \mt{val} \; X) &=& \overline{y ::: \mt{Type}} \to M.x \; \overline y \textrm{ (where $X \in \overline{dc}$)} \\
  \mt{proj}(M, \mt{datatype} \; x \; \overline{y} = \overline{dc} \; \overline{s}, \mt{val} \; X) &=& \overline{y ::: \mt{Type}} \to \tau \to M.x \; \overline y \textrm{ (where $X \; \mt{of} \; \tau \in \overline{dc}$)} \\
  \mt{proj}(M, \mt{datatype} \; x = \mt{datatype} \; M'.z, \mt{val} \; X) &=& \overline{y ::: \mt{Type}} \to M.x \; \overline y \textrm{ (where $\Gamma \vdash M' : \mt{sig} \; \overline{s'} \; \mt{end}$} \\
  && \textrm{and $\mt{proj}(M', \overline{s'}, \mt{datatype} \; z = (\overline{y}, \overline{dc})$ and $X \in \overline{dc}$)} \\
  \mt{proj}(M, \mt{datatype} \; x = \mt{datatype} \; M'.z, \mt{val} \; X) &=& \overline{y ::: \mt{Type}} \to \tau \to M.x \; \overline y \textrm{ (where $\Gamma \vdash M' : \mt{sig} \; \overline{s'} \; \mt{end}$} \\
  && \textrm{and $\mt{proj}(M', \overline{s'}, \mt{datatype} \; z = (\overline{y}, \overline{dc})$ and $X \; \mt{of} \; \tau \in \overline{dc}$)} \\
  \\
  \mt{proj}(M, \mt{structure} \; X : S \; \overline{s}, \mt{structure} \; X) &=& S \\
  \\
  \mt{proj}(M, \mt{signature} \; X = S \; \overline{s}, \mt{signature} \; X) &=& S \\
  \\
  \mt{proj}(M, \mt{con} \; x :: \kappa \; \overline{s}, V) &=& [x \mapsto M.x]\mt{proj}(M, \overline{s}, V) \\
  \mt{proj}(M, \mt{con} \; x :: \kappa = c \; \overline{s}, V) &=& [x \mapsto M.x]\mt{proj}(M, \overline{s}, V) \\
  \mt{proj}(M, \mt{datatype} \; x \; \overline{y} = \overline{dc} \; \overline{s}, V) &=& [x \mapsto M.x]\mt{proj}(M, \overline{s}, V) \\
  \mt{proj}(M, \mt{datatype} \; x = \mt{datatype} \; M'.z \; \overline{s}, V) &=& [x \mapsto M.x]\mt{proj}(M, \overline{s}, V) \\
  \mt{proj}(M, \mt{val} \; x : \tau \; \overline{s}, V) &=& \mt{proj}(M, \overline{s}, V) \\
  \mt{proj}(M, \mt{structure} \; X : S \; \overline{s}, V) &=& [X \mapsto M.X]\mt{proj}(M, \overline{s}, V) \\
  \mt{proj}(M, \mt{signature} \; X = S \; \overline{s}, V) &=& [X \mapsto M.X]\mt{proj}(M, \overline{s}, V) \\
  \mt{proj}(M, \mt{include} \; S \; \overline{s}, V) &=& \mt{proj}(M, \overline{s'} \; \overline{s}, V) \textrm{ (where $\Gamma \vdash S \equiv \mt{sig} \; \overline{s'} \; \mt{end}$)} \\
  \mt{proj}(M, \mt{constraint} \; c_1 \sim c_2 \; \overline{s}, V) &=& \mt{proj}(M, \overline{s}, V) \\
  \mt{proj}(M, \mt{class} \; x \; \overline{s}, V) &=& [x \mapsto M.x]\mt{proj}(M, \overline{s}, V) \\
  \mt{proj}(M, \mt{class} \; x = c \; \overline{s}, V) &=& [x \mapsto M.x]\mt{proj}(M, \overline{s}, V) \\
\end{eqnarray*}


\section{Type Inference}

The Ur/Web compiler uses \emph{heuristic type inference}, with no claims of completeness with respect to the declarative specification of the last section.  The rules in use seem to work well in practice.  This section summarizes those rules, to help Ur programmers predict what will work and what won't.

\subsection{Basic Unification}

Type-checkers for languages based on the Hindly-Milner type discipline, like ML and Haskell, take advantage of \emph{principal typing} properties, making complete type inference relatively straightforward.  Inference algorithms are traditionally implemented using type unification variables, at various points asserting equalities between types, in the process discovering the values of type variables.  The Ur/Web compiler uses the same basic strategy, but the complexity of the type system rules out easy completeness.

Type-checking can require evaluating recursive functional programs, thanks to the type-level $\mt{fold}$ operator.  When a unification variable appears in such a type, the next step of computation can be undetermined.  The value of that variable might be determined later, but this would be ``too late'' for the unification problems generated at the first occurrence.  This is the essential source of incompletness.

Nonetheless, the unification engine tends to do reasonably well.  Unlike in ML, polymorphism is never inferred in definitions; it must be indicated explicitly by writing out constructor-level parameters.  By writing these and other annotations, the programmer can generally get the type inference engine to do most of the type reconstruction work.

\subsection{Unifying Record Types}

The type inference engine tries to take advantage of the algebraic rules governing type-level records, as shown in Section \ref{definitional}.  When two constructors of record kind are unified, they are reduce to normal forms, with like terms crossed off from each normal form until, hopefully, nothing remains.  This cannot be complete, with the inclusion of unification variables.  The type-checker can help you understand what goes wrong when the process fails, as it outputs the unmatched remainders of the two normal forms.

\subsection{\label{typeclasses}Type Classes}

Ur includes a type class facility inspired by Haskell's.  The current version is very rudimentary, only supporting instances for particular types built up from abstract types and datatypes and type-level application.

Type classes are integrated with the module system.  A type class is just a constructor of kind $\mt{Type} \to \mt{Type}$.  By marking such a constructor $c$ as a type class, the programmer instructs the type inference engine to, in each scope, record all values of types $c \; \tau$ as \emph{instances}.  Any function argument whose type is of such a form is treated as implicit, to be determined by examining the current instance database.

The ``dictionary encoding'' often used in Haskell implementations is made explicit in Ur.  Type class instances are just properly-typed values, and they can also be considered as ``proofs'' of membership in the class.  In some cases, it is useful to pass these proofs around explicitly.  An underscore written where a proof is expected will also be inferred, if possible, from the current instance database.

Just as for types, type classes may be exported from modules, and they may be exported as concrete or abstract.  Concrete type classes have their ``real'' definitions exposed, so that client code may add new instances freely.  Abstract type classes are useful as ``predicates'' that can be used to enforce invariants, as we will see in some definitions of SQL syntax in the Ur/Web standard library.

\subsection{Reverse-Engineering Record Types}

It's useful to write Ur functions and functors that take record constructors as inputs, but these constructors can grow quite long, even though their values are often implied by other arguments.  The compiler uses a simple heuristic to infer the values of unification variables that are folded over, yielding known results.  Often, as in the case of $\mt{map}$-like folds, the base and recursive cases of a fold produce constructors with different top-level structure.  Thus, if the result of the fold is known, examining its top-level structure reveals whether the record being folded over is empty or not.  If it's empty, we're done; if it's not empty, we replace a single unification variable with a new constructor formed from three new unification variables, as in $[\alpha = \beta] \rc \gamma$.  This process can often be repeated to determine a unification variable fully.

\subsection{Implicit Arguments in Functor Applications}

Constructor, constraint, and type class witness members of structures may be omitted, when those structures are used in contexts where their assigned signatures imply how to fill in those missing members.  This feature combines well with reverse-engineering to allow for uses of complicated meta-programming functors with little more code than would be necessary to invoke an untyped, ad-hoc code generator.


\section{The Ur Standard Library}

The built-in parts of the Ur/Web standard library are described by the signature in \texttt{lib/basis.urs} in the distribution.  A module $\mt{Basis}$ ascribing to that signature is available in the initial environment, and every program is implicitly prefixed by $\mt{open} \; \mt{Basis}$.

Additionally, other common functions that are definable within Ur are included in \texttt{lib/top.urs} and \texttt{lib/top.ur}.  This $\mt{Top}$ module is also opened implicitly.

The idea behind Ur is to serve as the ideal host for embedded domain-specific languages.  For now, however, the ``generic'' functionality is intermixed with Ur/Web-specific functionality, including in these two library modules.  We hope that these generic library components have types that speak for themselves.  The next section introduces the Ur/Web-specific elements.  Here, we only give the type declarations from the beginning of $\mt{Basis}$.
$$\begin{array}{l}
  \mt{type} \; \mt{int} \\
  \mt{type} \; \mt{float} \\
  \mt{type} \; \mt{string} \\
  \mt{type} \; \mt{time} \\
  \\
  \mt{type} \; \mt{unit} = \{\} \\
  \\
  \mt{datatype} \; \mt{bool} = \mt{False} \mid \mt{True} \\
  \\
  \mt{datatype} \; \mt{option} \; \mt{t} = \mt{None} \mid \mt{Some} \; \mt{of} \; \mt{t}
\end{array}$$


\section{The Ur/Web Standard Library}

\subsection{Transactions}

Ur is a pure language; we use Haskell's trick to support controlled side effects.  The standard library defines a monad $\mt{transaction}$, meant to stand for actions that may be undone cleanly.  By design, no other kinds of actions are supported.
$$\begin{array}{l}
  \mt{con} \; \mt{transaction} :: \mt{Type} \to \mt{Type} \\
  \\
  \mt{val} \; \mt{return} : \mt{t} ::: \mt{Type} \to \mt{t} \to \mt{transaction} \; \mt{t} \\
  \mt{val} \; \mt{bind} : \mt{t_1} ::: \mt{Type} \to \mt{t_2} ::: \mt{Type} \to \mt{transaction} \; \mt{t_1} \to (\mt{t_1} \to \mt{transaction} \; \mt{t_2}) \to \mt{transaction} \; \mt{t_2}
\end{array}$$

\subsection{HTTP}

There are transactions for reading an HTTP header by name and for getting and setting strongly-typed cookies.  Cookies may only be created by the $\mt{cookie}$ declaration form, ensuring that they be named consistently based on module structure.
$$\begin{array}{l}
\mt{val} \; \mt{requestHeader} : \mt{string} \to \mt{transaction} \; (\mt{option} \; \mt{string}) \\
\\
\mt{con} \; \mt{http\_cookie} :: \mt{Type} \to \mt{Type} \\
\mt{val} \; \mt{getCookie} : \mt{t} ::: \mt{Type} \to \mt{http\_cookie} \; \mt{t} \to \mt{transaction} \; (\mt{option} \; \mt{t}) \\
\mt{val} \; \mt{setCookie} : \mt{t} ::: \mt{Type} \to \mt{http\_cookie} \; \mt{t} \to \mt{t} \to \mt{transaction} \; \mt{unit}
\end{array}$$

\subsection{SQL}

The fundamental unit of interest in the embedding of SQL is tables, described by a type family and creatable only via the $\mt{table}$ declaration form.
$$\begin{array}{l}
  \mt{con} \; \mt{sql\_table} :: \{\mt{Type}\} \to \mt{Type}
\end{array}$$

\subsubsection{Queries}

A final query is constructed via the $\mt{sql\_query}$ function.  Constructor arguments respectively specify the table fields we select (as records mapping tables to the subsets of their fields that we choose) and the (always named) extra expressions that we select.
$$\begin{array}{l}
  \mt{con} \; \mt{sql\_query} :: \{\{\mt{Type}\}\} \to \{\mt{Type}\} \to \mt{Type} \\
  \mt{val} \; \mt{sql\_query} : \mt{tables} ::: \{\{\mt{Type}\}\} \\
  \hspace{.1in} \to \mt{selectedFields} ::: \{\{\mt{Type}\}\} \\
  \hspace{.1in} \to \mt{selectedExps} ::: \{\mt{Type}\} \\
  \hspace{.1in} \to \{\mt{Rows} : \mt{sql\_query1} \; \mt{tables} \; \mt{selectedFields} \; \mt{selectedExps}, \\
  \hspace{.2in} \mt{OrderBy} : \mt{sql\_order\_by} \; \mt{tables} \; \mt{selectedExps}, \\
  \hspace{.2in} \mt{Limit} : \mt{sql\_limit}, \\
  \hspace{.2in} \mt{Offset} : \mt{sql\_offset}\} \\
  \hspace{.1in} \to \mt{sql\_query} \; \mt{selectedFields} \; \mt{selectedExps}
\end{array}$$

Queries are used by folding over their results inside transactions.
$$\begin{array}{l}
  \mt{val} \; \mt{query} : \mt{tables} ::: \{\{\mt{Type}\}\} \to \mt{exps} ::: \{\mt{Type}\} \to \lambda [\mt{tables} \sim \mt{exps}] \Rightarrow \mt{state} ::: \mt{Type} \to \mt{sql\_query} \; \mt{tables} \; \mt{exps} \\
  \hspace{.1in} \to (\$(\mt{exps} \rc \mt{fold} \; (\lambda \mt{nm} \; (\mt{fields} :: \{\mt{Type}\}) \; \mt{acc} \; [[\mt{nm}] \sim \mt{acc}] \Rightarrow [\mt{nm} = \$\mt{fields}] \rc \mt{acc}) \; [] \; \mt{tables}) \\
  \hspace{.2in} \to \mt{state} \to \mt{transaction} \; \mt{state}) \\
  \hspace{.1in} \to \mt{state} \to \mt{transaction} \; \mt{state}
\end{array}$$

Most of the complexity of the query encoding is in the type $\mt{sql\_query1}$, which includes simple queries and derived queries based on relational operators.  Constructor arguments respectively specify the tables we select from, the subset of fields that we keep from each table for the result rows, and the extra expressions that we select.
$$\begin{array}{l}
  \mt{con} \; \mt{sql\_query1} :: \{\{\mt{Type}\}\} \to \{\{\mt{Type}\}\} \to \{\mt{Type}\} \to \mt{Type} \\
  \\
  \mt{type} \; \mt{sql\_relop} \\
  \mt{val} \; \mt{sql\_union} : \mt{sql\_relop} \\
  \mt{val} \; \mt{sql\_intersect} : \mt{sql\_relop} \\
  \mt{val} \; \mt{sql\_except} : \mt{sql\_relop} \\
  \mt{val} \; \mt{sql\_relop} : \mt{tables1} ::: \{\{\mt{Type}\}\} \\
  \hspace{.1in} \to \mt{tables2} ::: \{\{\mt{Type}\}\} \\
  \hspace{.1in} \to \mt{selectedFields} ::: \{\{\mt{Type}\}\} \\
  \hspace{.1in} \to \mt{selectedExps} ::: \{\mt{Type}\} \\
  \hspace{.1in} \to \mt{sql\_relop} \\
  \hspace{.1in} \to \mt{sql\_query1} \; \mt{tables1} \; \mt{selectedFields} \; \mt{selectedExps} \\
  \hspace{.1in} \to \mt{sql\_query1} \; \mt{tables2} \; \mt{selectedFields} \; \mt{selectedExps} \\
  \hspace{.1in} \to \mt{sql\_query1} \; \mt{selectedFields} \; \mt{selectedFields} \; \mt{selectedExps}
\end{array}$$

$$\begin{array}{l}
  \mt{val} \; \mt{sql\_query1} : \mt{tables} ::: \{\{\mt{Type}\}\} \\
  \hspace{.1in} \to \mt{grouped} ::: \{\{\mt{Type}\}\} \\
  \hspace{.1in} \to \mt{selectedFields} ::: \{\{\mt{Type}\}\} \\
  \hspace{.1in} \to \mt{selectedExps} ::: \{\mt{Type}\} \\
  \hspace{.1in} \to \{\mt{From} : \$(\mt{fold} \; (\lambda \mt{nm} \; (\mt{fields} :: \{\mt{Type}\}) \; \mt{acc} \; [[\mt{nm}] \sim \mt{acc}] \Rightarrow [\mt{nm} = \mt{sql\_table} \; \mt{fields}] \rc \mt{acc}) \; [] \; \mt{tables}), \\
  \hspace{.2in} \mt{Where} : \mt{sql\_exp} \; \mt{tables} \; [] \; [] \; \mt{bool}, \\
  \hspace{.2in} \mt{GroupBy} : \mt{sql\_subset} \; \mt{tables} \; \mt{grouped}, \\
  \hspace{.2in} \mt{Having} : \mt{sql\_exp} \; \mt{grouped} \; \mt{tables} \; [] \; \mt{bool}, \\
  \hspace{.2in} \mt{SelectFields} : \mt{sql\_subset} \; \mt{grouped} \; \mt{selectedFields}, \\
  \hspace{.2in} \mt {SelectExps} : \$(\mt{fold} \; (\lambda \mt{nm} \; (\mt{t} :: \mt{Type}) \; \mt{acc} \; [[\mt{nm}] \sim \mt{acc}] \Rightarrow [\mt{nm} = \mt{sql\_exp} \; \mt{grouped} \; \mt{tables} \; [] \; \mt{t}] \rc \mt{acc}) \; [] \; \mt{selectedExps}) \} \\
  \hspace{.1in} \to \mt{sql\_query1} \; \mt{tables} \; \mt{selectedFields} \; \mt{selectedExps}
\end{array}$$

To encode projection of subsets of fields in $\mt{SELECT}$ clauses, and to encode $\mt{GROUP} \; \mt{BY}$ clauses, we rely on a type family $\mt{sql\_subset}$, capturing what it means for one record of table fields to be a subset of another.  The main constructor $\mt{sql\_subset}$ ``proves subset facts'' by requiring a split of a record into kept and dropped parts.  The extra constructor $\mt{sql\_subset\_all}$ is a convenience for keeping all fields of a record.
$$\begin{array}{l}
  \mt{con} \; \mt{sql\_subset} :: \{\{\mt{Type}\}\} \to \{\{\mt{Type}\}\} \to \mt{Type} \\
  \mt{val} \; \mt{sql\_subset} : \mt{keep\_drop} :: \{(\{\mt{Type}\} \times \{\mt{Type}\})\} \\
  \hspace{.1in} \to \mt{sql\_subset} \\
  \hspace{.2in} (\mt{fold} \; (\lambda \mt{nm} \; (\mt{fields} :: (\{\mt{Type}\} \times \{\mt{Type}\})) \; \mt{acc} \; [[\mt{nm}] \sim \mt{acc}] \; [\mt{fields}.1 \sim \mt{fields}.2] \Rightarrow \\
  \hspace{.3in} [\mt{nm} = \mt{fields}.1 \rc \mt{fields}.2] \rc \mt{acc}) \; [] \; \mt{keep\_drop}) \\
  \hspace{.2in} (\mt{fold} \; (\lambda \mt{nm} \; (\mt{fields} :: (\{\mt{Type}\} \times \{\mt{Type}\})) \; \mt{acc} \; [[\mt{nm}] \sim \mt{acc}] \Rightarrow [\mt{nm} = \mt{fields}.1] \rc \mt{acc}) \; [] \; \mt{keep\_drop}) \\
\mt{val} \; \mt{sql\_subset\_all} : \mt{tables} :: \{\{\mt{Type}\}\} \to \mt{sql\_subset} \; \mt{tables} \; \mt{tables}
\end{array}$$

SQL expressions are used in several places, including $\mt{SELECT}$, $\mt{WHERE}$, $\mt{HAVING}$, and $\mt{ORDER} \; \mt{BY}$ clauses.  They reify a fragment of the standard SQL expression language, while making it possible to inject ``native'' Ur values in some places.  The arguments to the $\mt{sql\_exp}$ type family respectively give the unrestricted-availablity table fields, the table fields that may only be used in arguments to aggregate functions, the available selected expressions, and the type of the expression.
$$\begin{array}{l}
  \mt{con} \; \mt{sql\_exp} :: \{\{\mt{Type}\}\} \to \{\{\mt{Type}\}\} \to \{\mt{Type}\} \to \mt{Type} \to \mt{Type}
\end{array}$$

Any field in scope may be converted to an expression.
$$\begin{array}{l}
  \mt{val} \; \mt{sql\_field} : \mt{otherTabs} ::: \{\{\mt{Type}\}\} \to \mt{otherFields} ::: \{\mt{Type}\} \\
  \hspace{.1in} \to \mt{fieldType} ::: \mt{Type} \to \mt{agg} ::: \{\{\mt{Type}\}\} \\
  \hspace{.1in} \to \mt{exps} ::: \{\mt{Type}\} \\
  \hspace{.1in} \to \mt{tab} :: \mt{Name} \to \mt{field} :: \mt{Name} \\
  \hspace{.1in} \to \mt{sql\_exp} \; ([\mt{tab} = [\mt{field} = \mt{fieldType}] \rc \mt{otherFields}] \rc \mt{otherTabs}) \; \mt{agg} \; \mt{exps} \; \mt{fieldType}
\end{array}$$

There is an analogous function for referencing named expressions.
$$\begin{array}{l}
  \mt{val} \; \mt{sql\_exp} : \mt{tabs} ::: \{\{\mt{Type}\}\} \to \mt{agg} ::: \{\{\mt{Type}\}\} \to \mt{t} ::: \mt{Type} \to \mt{rest} ::: \{\mt{Type}\} \to \mt{nm} :: \mt{Name} \\
  \hspace{.1in} \to \mt{sql\_exp} \; \mt{tabs} \; \mt{agg} \; ([\mt{nm} = \mt{t}] \rc \mt{rest}) \; \mt{t}
\end{array}$$

Ur values of appropriate types may be injected into SQL expressions.
$$\begin{array}{l}
  \mt{class} \; \mt{sql\_injectable} \\
  \mt{val} \; \mt{sql\_bool} : \mt{sql\_injectable} \; \mt{bool} \\
  \mt{val} \; \mt{sql\_int} : \mt{sql\_injectable} \; \mt{int} \\
  \mt{val} \; \mt{sql\_float} : \mt{sql\_injectable} \; \mt{float} \\
  \mt{val} \; \mt{sql\_string} : \mt{sql\_injectable} \; \mt{string} \\
  \mt{val} \; \mt{sql\_time} : \mt{sql\_injectable} \; \mt{time} \\
  \mt{val} \; \mt{sql\_option\_bool} : \mt{sql\_injectable} \; (\mt{option} \; \mt{bool}) \\
  \mt{val} \; \mt{sql\_option\_int} : \mt{sql\_injectable} \; (\mt{option} \; \mt{int}) \\
  \mt{val} \; \mt{sql\_option\_float} : \mt{sql\_injectable} \; (\mt{option} \; \mt{float}) \\
  \mt{val} \; \mt{sql\_option\_string} : \mt{sql\_injectable} \; (\mt{option} \; \mt{string}) \\
  \mt{val} \; \mt{sql\_option\_time} : \mt{sql\_injectable} \; (\mt{option} \; \mt{time}) \\
  \mt{val} \; \mt{sql\_inject} : \mt{tables} ::: \{\{\mt{Type}\}\} \to \mt{agg} ::: \{\{\mt{Type}\}\} \to \mt{exps} ::: \{\mt{Type}\} \to \mt{t} ::: \mt{Type} \to \mt{sql\_injectable} \; \mt{t} \\
  \hspace{.1in} \to \mt{t} \to \mt{sql\_exp} \; \mt{tables} \; \mt{agg} \; \mt{exps} \; \mt{t}
\end{array}$$

We have the SQL nullness test, which is necessary because of the strange SQL semantics of equality in the presence of null values.
$$\begin{array}{l}
  \mt{val} \; \mt{sql\_is\_null} : \mt{tables} ::: \{\{\mt{Type}\}\} \to \mt{agg} ::: \{\{\mt{Type}\}\} \to \mt{exps} ::: \{\mt{Type}\} \to \mt{t} ::: \mt{Type} \\
  \hspace{.1in} \to \mt{sql\_exp} \; \mt{tables} \; \mt{agg} \; \mt{exps} \; (\mt{option} \; \mt{t}) \to \mt{sql\_exp} \; \mt{tables} \; \mt{agg} \; \mt{exps} \; \mt{bool}
\end{array}$$

We have generic nullary, unary, and binary operators, as well as comparison operators.
$$\begin{array}{l}
  \mt{con} \; \mt{sql\_nfunc} :: \mt{Type} \to \mt{Type} \\
  \mt{val} \; \mt{sql\_current\_timestamp} : \mt{sql\_nfunc} \; \mt{time} \\
  \mt{val} \; \mt{sql\_nfunc} : \mt{tables} ::: \{\{\mt{Type}\}\} \to \mt{agg} ::: \{\{\mt{Type}\}\} \to \mt{exps} ::: \{\mt{Type}\} \to \mt{t} ::: \mt{Type} \\
  \hspace{.1in} \to \mt{sql\_nfunc} \; \mt{t} \to \mt{sql\_exp} \; \mt{tables} \; \mt{agg} \; \mt{exps} \; \mt{t} \\\end{array}$$

$$\begin{array}{l}
  \mt{con} \; \mt{sql\_unary} :: \mt{Type} \to \mt{Type} \to \mt{Type} \\
  \mt{val} \; \mt{sql\_not} : \mt{sql\_unary} \; \mt{bool} \; \mt{bool} \\
  \mt{val} \; \mt{sql\_unary} : \mt{tables} ::: \{\{\mt{Type}\}\} \to \mt{agg} ::: \{\{\mt{Type}\}\} \to \mt{exps} ::: \{\mt{Type}\} \to \mt{arg} ::: \mt{Type} \to \mt{res} ::: \mt{Type} \\
  \hspace{.1in} \to \mt{sql\_unary} \; \mt{arg} \; \mt{res} \to \mt{sql\_exp} \; \mt{tables} \; \mt{agg} \; \mt{exps} \; \mt{arg} \to \mt{sql\_exp} \; \mt{tables} \; \mt{agg} \; \mt{exps} \; \mt{res} \\
\end{array}$$

$$\begin{array}{l}
  \mt{con} \; \mt{sql\_binary} :: \mt{Type} \to \mt{Type} \to \mt{Type} \to \mt{Type} \\
  \mt{val} \; \mt{sql\_and} : \mt{sql\_binary} \; \mt{bool} \; \mt{bool} \; \mt{bool} \\
  \mt{val} \; \mt{sql\_or} : \mt{sql\_binary} \; \mt{bool} \; \mt{bool} \; \mt{bool} \\
  \mt{val} \; \mt{sql\_binary} : \mt{tables} ::: \{\{\mt{Type}\}\} \to \mt{agg} ::: \{\{\mt{Type}\}\} \to \mt{exps} ::: \{\mt{Type}\} \to \mt{arg_1} ::: \mt{Type} \to \mt{arg_2} ::: \mt{Type} \to \mt{res} ::: \mt{Type} \\
  \hspace{.1in} \to \mt{sql\_binary} \; \mt{arg_1} \; \mt{arg_2} \; \mt{res} \to \mt{sql\_exp} \; \mt{tables} \; \mt{agg} \; \mt{exps} \; \mt{arg_1} \to \mt{sql\_exp} \; \mt{tables} \; \mt{agg} \; \mt{exps} \; \mt{arg_2} \to \mt{sql\_exp} \; \mt{tables} \; \mt{agg} \; \mt{exps} \; \mt{res}
\end{array}$$

$$\begin{array}{l}
  \mt{type} \; \mt{sql\_comparison} \\
  \mt{val} \; \mt{sql\_eq} : \mt{sql\_comparison} \\
  \mt{val} \; \mt{sql\_ne} : \mt{sql\_comparison} \\
  \mt{val} \; \mt{sql\_lt} : \mt{sql\_comparison} \\
  \mt{val} \; \mt{sql\_le} : \mt{sql\_comparison} \\
  \mt{val} \; \mt{sql\_gt} : \mt{sql\_comparison} \\
  \mt{val} \; \mt{sql\_ge} : \mt{sql\_comparison} \\
  \mt{val} \; \mt{sql\_comparison} : \mt{tables} ::: \{\{\mt{Type}\}\} \to \mt{agg} ::: \{\{\mt{Type}\}\} \to \mt{exps} ::: \{\mt{Type}\} \to \mt{t} ::: \mt{Type} \\
  \hspace{.1in} \to \mt{sql\_comparison} \to \mt{sql\_exp} \; \mt{tables} \; \mt{agg} \; \mt{exps} \; \mt{t} \to \mt{sql\_exp} \; \mt{tables} \; \mt{agg} \; \mt{exps} \; \mt{t} \to \mt{sql\_exp} \; \mt{tables} \; \mt{agg} \; \mt{exps} \; \mt{bool}
 \end{array}$$

Finally, we have aggregate functions.  The $\mt{COUNT(\ast)}$ syntax is handled specially, since it takes no real argument.  The other aggregate functions are placed into a general type family, using type classes to restrict usage to properly-typed arguments.  The key aspect of the $\mt{sql\_aggregate}$ function's type is the shift of aggregate-function-only fields into unrestricted fields.

$$\begin{array}{l}
  \mt{val} \; \mt{sql\_count} : \mt{tables} ::: \{\{\mt{Type}\}\} \to \mt{agg} ::: \{\{\mt{Type}\}\} \to \mt{exps} ::: \{\mt{Type}\} \to \mt{sql\_exp} \; \mt{tables} \; \mt{agg} \; \mt{exps} \; \mt{int}
\end{array}$$

$$\begin{array}{l}
  \mt{con} \; \mt{sql\_aggregate} :: \mt{Type} \to \mt{Type} \\
  \mt{val} \; \mt{sql\_aggregate} : \mt{tables} ::: \{\{\mt{Type}\}\} \to \mt{agg} ::: \{\{\mt{Type}\}\} \to \mt{exps} ::: \{\mt{Type}\} \to \mt{t} ::: \mt{Type} \\
  \hspace{.1in} \to \mt{sql\_aggregate} \; \mt{t} \to \mt{sql\_exp} \; \mt{agg} \; \mt{agg} \; \mt{exps} \; \mt{t} \to \mt{sql\_exp} \; \mt{tables} \; \mt{agg} \; \mt{exps} \; \mt{t}
\end{array}$$
 
$$\begin{array}{l}
  \mt{class} \; \mt{sql\_summable} \\
  \mt{val} \; \mt{sql\_summable\_int} : \mt{sql\_summable} \; \mt{int} \\
  \mt{val} \; \mt{sql\_summable\_float} : \mt{sql\_summable} \; \mt{float} \\
  \mt{val} \; \mt{sql\_avg} : \mt{t} ::: \mt{Type} \to \mt{sql\_summable} \; \mt{t} \to \mt{sql\_aggregate} \; \mt{t} \\
  \mt{val} \; \mt{sql\_sum} : \mt{t} ::: \mt{Type} \to \mt{sql\_summable} \mt{t} \to \mt{sql\_aggregate} \; \mt{t}
\end{array}$$

$$\begin{array}{l}
  \mt{class} \; \mt{sql\_maxable} \\
  \mt{val} \; \mt{sql\_maxable\_int} : \mt{sql\_maxable} \; \mt{int} \\
  \mt{val} \; \mt{sql\_maxable\_float} : \mt{sql\_maxable} \; \mt{float} \\
  \mt{val} \; \mt{sql\_maxable\_string} : \mt{sql\_maxable} \; \mt{string} \\
  \mt{val} \; \mt{sql\_maxable\_time} : \mt{sql\_maxable} \; \mt{time} \\
  \mt{val} \; \mt{sql\_max} : \mt{t} ::: \mt{Type} \to \mt{sql\_maxable} \; \mt{t} \to \mt{sql\_aggregate} \; \mt{t} \\
  \mt{val} \; \mt{sql\_min} : \mt{t} ::: \mt{Type} \to \mt{sql\_maxable} \; \mt{t} \to \mt{sql\_aggregate} \; \mt{t}
\end{array}$$

We wrap up the definition of query syntax with the types used in representing $\mt{ORDER} \; \mt{BY}$, $\mt{LIMIT}$, and $\mt{OFFSET}$ clauses.
$$\begin{array}{l}
  \mt{type} \; \mt{sql\_direction} \\
  \mt{val} \; \mt{sql\_asc} : \mt{sql\_direction} \\
  \mt{val} \; \mt{sql\_desc} : \mt{sql\_direction} \\
  \\
  \mt{con} \; \mt{sql\_order\_by} :: \{\{\mt{Type}\}\} \to \{\mt{Type}\} \to \mt{Type} \\
  \mt{val} \; \mt{sql\_order\_by\_Nil} : \mt{tables} ::: \{\{\mt{Type}\}\} \to \mt{exps} :: \{\mt{Type}\} \to \mt{sql\_order\_by} \; \mt{tables} \; \mt{exps} \\
  \mt{val} \; \mt{sql\_order\_by\_Cons} : \mt{tables} ::: \{\{\mt{Type}\}\} \to \mt{exps} ::: \{\mt{Type}\} \to \mt{t} ::: \mt{Type} \\
  \hspace{.1in} \to \mt{sql\_exp} \; \mt{tables} \; [] \; \mt{exps} \; \mt{t} \to \mt{sql\_direction} \to \mt{sql\_order\_by} \; \mt{tables} \; \mt{exps} \to \mt{sql\_order\_by} \; \mt{tables} \; \mt{exps} \\
  \\
  \mt{type} \; \mt{sql\_limit} \\
  \mt{val} \; \mt{sql\_no\_limit} : \mt{sql\_limit} \\
  \mt{val} \; \mt{sql\_limit} : \mt{int} \to \mt{sql\_limit} \\
  \\
  \mt{type} \; \mt{sql\_offset} \\
  \mt{val} \; \mt{sql\_no\_offset} : \mt{sql\_offset} \\
  \mt{val} \; \mt{sql\_offset} : \mt{int} \to \mt{sql\_offset}
\end{array}$$


\subsubsection{DML}

The Ur/Web library also includes an embedding of a fragment of SQL's DML, the Data Manipulation Language, for modifying database tables.  Any piece of DML may be executed in a transaction.

$$\begin{array}{l}
  \mt{type} \; \mt{dml} \\
  \mt{val} \; \mt{dml} : \mt{dml} \to \mt{transaction} \; \mt{unit}
\end{array}$$

Properly-typed records may be used to form $\mt{INSERT}$ commands.
$$\begin{array}{l}
  \mt{val} \; \mt{insert} : \mt{fields} ::: \{\mt{Type}\} \to \mt{sql\_table} \; \mt{fields} \\
  \hspace{.1in} \to \$(\mt{fold} \; (\lambda \mt{nm} \; (\mt{t} :: \mt{Type}) \; \mt{acc} \; [[\mt{nm}] \sim \mt{acc}] \Rightarrow [\mt{nm} = \mt{sql\_exp} \; [] \; [] \; [] \; \mt{t}] \rc \mt{acc}) \; [] \; \mt{fields}) \to \mt{dml}
\end{array}$$

An $\mt{UPDATE}$ command is formed from a choice of which table fields to leave alone and which to change, along with an expression to use to compute the new value of each changed field and a $\mt{WHERE}$ clause.
$$\begin{array}{l}
  \mt{val} \; \mt{update} : \mt{unchanged} ::: \{\mt{Type}\} \to \mt{changed} :: \{\mt{Type}\} \to \lambda [\mt{changed} \sim \mt{unchanged}] \\
  \hspace{.1in} \Rightarrow \$(\mt{fold} \; (\lambda \mt{nm} \; (\mt{t} :: \mt{Type}) \; \mt{acc} \; [[\mt{nm}] \sim \mt{acc}] \Rightarrow [\mt{nm} = \mt{sql\_exp} \; [\mt{T} = \mt{changed} \rc \mt{unchanged}] \; [] \; [] \; \mt{t}] \rc \mt{acc}) \; [] \; \mt{changed}) \\
  \hspace{.1in} \to \mt{sql\_table} \; (\mt{changed} \rc \mt{unchanged}) \to \mt{sql\_exp} \; [\mt{T} = \mt{changed} \rc \mt{unchanged}] \; [] \; [] \; \mt{bool} \to \mt{dml}
\end{array}$$

A $\mt{DELETE}$ command is formed from a table and a $\mt{WHERE}$ clause.
$$\begin{array}{l}
  \mt{val} \; \mt{delete} : \mt{fields} ::: \{\mt{Type}\} \to \mt{sql\_table} \; \mt{fields} \to \mt{sql\_exp} \; [\mt{T} = \mt{fields}] \; [] \; [] \; \mt{bool} \to \mt{dml}
\end{array}$$

\subsubsection{Sequences}

SQL sequences are counters with concurrency control, often used to assign unique IDs.  Ur/Web supports them via a simple interface.  The only way to create a sequence is with the $\mt{sequence}$ declaration form.

$$\begin{array}{l}
  \mt{type} \; \mt{sql\_sequence} \\
  \mt{val} \; \mt{nextval} : \mt{sql\_sequence} \to \mt{transaction} \; \mt{int}
\end{array}$$


\subsection{XML}

Ur/Web's library contains an encoding of XML syntax and semantic constraints.  We make no effort to follow the standards governing XML schemas.  Rather, XML fragments are viewed more as values of ML datatypes, and we only track which tags are allowed inside which other tags.

The basic XML type family has arguments respectively indicating the \emph{context} of a fragment, the fields that the fragment expects to be bound on entry (and their types), and the fields that the fragment will bind (and their types).  Contexts are a record-based ``poor man's subtyping'' encoding, with each possible set of valid tags corresponding to a different context record.  The arguments dealing with field binding are only relevant to HTML forms.
$$\begin{array}{l}
  \mt{con} \; \mt{xml} :: \{\mt{Unit}\} \to \{\mt{Type}\} \to \{\mt{Type}\} \to \mt{Type}
\end{array}$$

We also have a type family of XML tags, indexed respectively by the record of optional attributes accepted by the tag, the context in which the tag may be placed, the context required of children of the tag, which form fields the tag uses, and which fields the tag defines.
$$\begin{array}{l}
  \mt{con} \; \mt{tag} :: \{\mt{Type}\} \to \{\mt{Unit}\} \to \{\mt{Unit}\} \to \{\mt{Type}\} \to \{\mt{Type}\} \to \mt{Type}
\end{array}$$

Literal text may be injected into XML as ``CDATA.''
$$\begin{array}{l}
  \mt{val} \; \mt{cdata} : \mt{ctx} ::: \{\mt{Unit}\} \to \mt{use} ::: \{\mt{Type}\} \to \mt{string} \to \mt{xml} \; \mt{ctx} \; \mt{use} \; []
\end{array}$$

There is a function for producing an XML tree with a particular tag at its root.
$$\begin{array}{l}
  \mt{val} \; \mt{tag} : \mt{attrsGiven} ::: \{\mt{Type}\} \to \mt{attrsAbsent} ::: \{\mt{Type}\} \to \mt{ctxOuter} ::: \{\mt{Unit}\} \to \mt{ctxInner} ::: \{\mt{Unit}\} \\
  \hspace{.1in} \to \mt{useOuter} ::: \{\mt{Type}\} \to \mt{useInner} ::: \{\mt{Type}\} \to \mt{bindOuter} ::: \{\mt{Type}\} \to \mt{bindInner} ::: \{\mt{Type}\} \\
  \hspace{.1in} \to \lambda [\mt{attrsGiven} \sim \mt{attrsAbsent}] \; [\mt{useOuter} \sim \mt{useInner}] \; [\mt{bindOuter} \sim \mt{bindInner}] \Rightarrow \$\mt{attrsGiven} \\
  \hspace{.1in} \to \mt{tag} \; (\mt{attrsGiven} \rc \mt{attrsAbsent}) \; \mt{ctxOuter} \; \mt{ctxInner} \; \mt{useOuter} \; \mt{bindOuter} \\
  \hspace{.1in} \to \mt{xml} \; \mt{ctxInner} \; \mt{useInner} \; \mt{bindInner} \to \mt{xml} \; \mt{ctxOuter} \; (\mt{useOuter} \rc \mt{useInner}) \; (\mt{bindOuter} \rc \mt{bindInner})
\end{array}$$

Two XML fragments may be concatenated.
$$\begin{array}{l}
  \mt{val} \; \mt{join} : \mt{ctx} ::: \{\mt{Unit}\} \to \mt{use_1} ::: \{\mt{Type}\} \to \mt{bind_1} ::: \{\mt{Type}\} \to \mt{bind_2} ::: \{\mt{Type}\} \\
  \hspace{.1in} \to \lambda [\mt{use_1} \sim \mt{bind_1}] \; [\mt{bind_1} \sim \mt{bind_2}] \\
  \hspace{.1in} \Rightarrow \mt{xml} \; \mt{ctx} \; \mt{use_1} \; \mt{bind_1} \to \mt{xml} \; \mt{ctx} \; (\mt{use_1} \rc \mt{bind_1}) \; \mt{bind_2} \to \mt{xml} \; \mt{ctx} \; \mt{use_1} \; (\mt{bind_1} \rc \mt{bind_2})
\end{array}$$

Finally, any XML fragment may be updated to ``claim'' to use more form fields than it does.
$$\begin{array}{l}
  \mt{val} \; \mt{useMore} : \mt{ctx} ::: \{\mt{Unit}\} \to \mt{use_1} ::: \{\mt{Type}\} \to \mt{use_2} ::: \{\mt{Type}\} \to \mt{bind} ::: \{\mt{Type}\} \to \lambda [\mt{use_1} \sim \mt{use_2}] \\
  \hspace{.1in} \Rightarrow \mt{xml} \; \mt{ctx} \; \mt{use_1} \; \mt{bind} \to \mt{xml} \; \mt{ctx} \; (\mt{use_1} \rc \mt{use_2}) \; \mt{bind}
\end{array}$$

We will not list here the different HTML tags and related functions from the standard library.  They should be easy enough to understand from the code in \texttt{basis.urs}.  The set of tags in the library is not yet claimed to be complete for HTML standards.

One last useful function is for aborting any page generation, returning some XML as an error message.  This function takes the place of some uses of a general exception mechanism.
$$\begin{array}{l}
  \mt{val} \; \mt{error} : \mt{t} ::: \mt{Type} \to \mt{xml} \; [\mt{Body}] \; [] \; [] \to \mt{t}
\end{array}$$


\section{Ur/Web Syntax Extensions}

Ur/Web features some syntactic shorthands for building values using the functions from the last section.  This section sketches the grammar of those extensions.  We write spans of syntax inside brackets to indicate that they are optional.

\subsection{SQL}

\subsubsection{Queries}

Queries $Q$ are added to the rules for expressions $e$.

$$\begin{array}{rrcll}
  \textrm{Queries} & Q &::=& (q \; [\mt{ORDER} \; \mt{BY} \; (E \; [o],)^+] \; [\mt{LIMIT} \; N] \; [\mt{OFFSET} \; N]) \\
  \textrm{Pre-queries} & q &::=& \mt{SELECT} \; P \; \mt{FROM} \; T,^+ \; [\mt{WHERE} \; E] \; [\mt{GROUP} \; \mt{BY} \; p,^+] \; [\mt{HAVING} \; E] \\
  &&& \mid q \; R \; q \\
  \textrm{Relational operators} & R &::=& \mt{UNION} \mid \mt{INTERSECT} \mid \mt{EXCEPT}
\end{array}$$

$$\begin{array}{rrcll}
  \textrm{Projections} & P &::=& \ast & \textrm{all columns} \\
  &&& p,^+ & \textrm{particular columns} \\
  \textrm{Pre-projections} & p &::=& t.f & \textrm{one column from a table} \\
  &&& t.\{\{c\}\} & \textrm{a record of columns from a table (of kind $\{\mt{Type}\}$)} \\
  \textrm{Table names} & t &::=& x & \textrm{constant table name (automatically capitalized)} \\
  &&& X & \textrm{constant table name} \\
  &&& \{\{c\}\} & \textrm{computed table name (of kind $\mt{Name}$)} \\
  \textrm{Column names} & f &::=& X & \textrm{constant column name} \\
  &&& \{c\} & \textrm{computed column name (of kind $\mt{Name}$)} \\
  \textrm{Tables} & T &::=& x & \textrm{table variable, named locally by its own capitalization} \\
  &&& x \; \mt{AS} \; t & \textrm{table variable, with local name} \\
  &&& \{\{e\}\} \; \mt{AS} \; t & \textrm{computed table expression, with local name} \\
  \textrm{SQL expressions} & E &::=& p & \textrm{column references} \\
  &&& X & \textrm{named expression references} \\
  &&& \{\{e\}\} & \textrm{injected native Ur expressions} \\
  &&& \{e\} & \textrm{computed expressions, probably using $\mt{sql\_exp}$ directly} \\
  &&& \mt{TRUE} \mid \mt{FALSE} & \textrm{boolean constants} \\
  &&& \ell & \textrm{primitive type literals} \\
  &&& \mt{NULL} & \textrm{null value (injection of $\mt{None}$)} \\
  &&& E \; \mt{IS} \; \mt{NULL} & \textrm{nullness test} \\
  &&& n & \textrm{nullary operators} \\
  &&& u \; E & \textrm{unary operators} \\
  &&& E \; b \; E & \textrm{binary operators} \\
  &&& \mt{COUNT}(\ast) & \textrm{count number of rows} \\
  &&& a(E) & \textrm{other aggregate function} \\
  &&& (E) & \textrm{explicit precedence} \\
  \textrm{Nullary operators} & n &::=& \mt{CURRENT\_TIMESTAMP} \\
  \textrm{Unary operators} & u &::=& \mt{NOT} \\
  \textrm{Binary operators} & b &::=& \mt{AND} \mid \mt{OR} \mid \neq \mid < \mid \leq \mid > \mid \geq \\
  \textrm{Aggregate functions} & a &::=& \mt{AVG} \mid \mt{SUM} \mid \mt{MIN} \mid \mt{MAX} \\
  \textrm{Directions} & o &::=& \mt{ASC} \mid \mt{DESC} \\
  \textrm{SQL integer} & N &::=& n \mid \{e\} \\
\end{array}$$

Additionally, an SQL expression may be inserted into normal Ur code with the syntax $(\mt{SQL} \; E)$ or $(\mt{WHERE} \; E)$.

\subsubsection{DML}

DML commands $D$ are added to the rules for expressions $e$.

$$\begin{array}{rrcll}
  \textrm{Commands} & D &::=& (\mt{INSERT} \; \mt{INTO} \; T^E \; (f,^+) \; \mt{VALUES} \; (E,^+)) \\
  &&& (\mt{UPDATE} \; T^E \; \mt{SET} \; (f = E,)^+ \; \mt{WHERE} \; E) \\
  &&& (\mt{DELETE} \; \mt{FROM} \; T^E \; \mt{WHERE} \; E) \\
  \textrm{Table expressions} & T^E &::=& x \mid \{\{e\}\}
\end{array}$$

Inside $\mt{UPDATE}$ and $\mt{DELETE}$ commands, lone variables $X$ are interpreted as references to columns of the implicit table $\mt{T}$, rather than to named expressions.

\subsection{XML}

XML fragments $L$ are added to the rules for expressions $e$.

$$\begin{array}{rrcll}
  \textrm{XML fragments} & L &::=& \texttt{<xml/>} \mid \texttt{<xml>}l^*\texttt{</xml>} \\
  \textrm{XML pieces} & l &::=& \textrm{text} & \textrm{cdata} \\
  &&& \texttt{<}g\texttt{/>} & \textrm{tag with no children} \\
  &&& \texttt{<}g\texttt{>}l^*\texttt{</}x\texttt{>} & \textrm{tag with children} \\
  \textrm{Tag} & g &::=& h \; (x = v)^* \\
  \textrm{Tag head} & h &::=& x & \textrm{tag name} \\
  &&& h\{c\} & \textrm{constructor parameter} \\
  \textrm{Attribute value} & v &::=& \ell & \textrm{literal value} \\
  &&& \{e\} & \textrm{computed value} \\
\end{array}$$


\section{The Structure of Web Applications}

A web application is built from a series of modules, with one module, the last one appearing in the \texttt{.urp} file, designated as the main module.  The signature of the main module determines the URL entry points to the application.  Such an entry point should have type $\mt{unit} \to \mt{transaction} \; \mt{page}$, where $\mt{page}$ is a type synonym for top-level HTML pages, defined in $\mt{Basis}$.  If such a function is at the top level of main module $M$, it will be accessible at URI \texttt{/M/f}, and so on for more deeply-nested functions, as described in Section \ref{tag} below.

When the standalone web server receives a request for a known page, it calls the function for that page, ``running'' the resulting transaction to produce the page to return to the client.  Pages link to other pages with the \texttt{link} attribute of the \texttt{a} HTML tag.  A link has type $\mt{transaction} \; \mt{page}$, and the semantics of a link are that this transaction should be run to compute the result page, when the link is followed.  Link targets are assigned URL names in the same way as top-level entry points.

HTML forms are handled in a similar way.  The $\mt{action}$ attribute of a $\mt{submit}$ form tag takes a value of type $\$\mt{use} \to \mt{transaction} \; \mt{page}$, where $\mt{use}$ is a kind-$\{\mt{Type}\}$ record of the form fields used by this action handler.  Action handlers are assigned URL patterns in the same way as above.

For both links and actions, direct arguments and local variables mentioned implicitly via closures are automatically included in serialized form in URLs, in the order in which they appear in the source code.


\section{Compiler Phases}

The Ur/Web compiler is unconventional in that it relies on a kind of \emph{heuristic compilation}.  Not all valid programs will compile successfully.  Informally, programs fail to compile when they are ``too higher order.''  Compiler phases do their best to eliminate different kinds of higher order-ness, but some programs just won't compile.  This is a trade-off for producing very efficient executables.  Compiled Ur/Web programs use native C representations and require no garbage collection.

In this section, we step through the main phases of compilation, noting what consequences each phase has for effective programming.

\subsection{Parse}

The compiler reads a \texttt{.urp} file, figures out which \texttt{.urs} and \texttt{.ur} files it references, and combines them all into what is conceptually a single sequence of declarations in the core language of Section \ref{core}.

\subsection{Elaborate}

This is where type inference takes place, translating programs into an explicit form with no more wildcards.  This phase is the most likely source of compiler error messages.

\subsection{Unnest}

Named local function definitions are moved to the top level, to avoid the need to generate closures.

\subsection{Corify}

Module system features are compiled away, through inlining of functor definitions at application sites.  Afterward, most abstraction boundaries are broken, facilitating optimization.

\subsection{Especialize}

Functions are specialized to particular argument patterns.  This is an important trick for avoiding the need to maintain any closures at runtime.

\subsection{Untangle}

Remove unnecessary mutual recursion, splitting recursive groups into strongly-connected components.

\subsection{Shake}

Remove all definitions not needed to run the page handlers that are visible in the signature of the last module listed in the \texttt{.urp} file.

\subsection{\label{tag}Tag}

Assign a URL name to each link and form action.  It is important that these links and actions are written as applications of named functions, because such names are used to generate URL patterns.  A URL pattern has a name built from the full module path of the named function, followed by the function name, with all pieces separated by slashes.  The path of a functor application is based on the name given to the result, rather than the path of the functor itself.

\subsection{Reduce}

Apply definitional equality rules to simplify the program as much as possible.  This effectively includes inlining of every non-recursive definition.

\subsection{Unpoly}

This phase specializes polymorphic functions to the specific arguments passed to them in the program.  If the program contains real polymorphic recursion, Unpoly will be insufficient to avoid later error messages about too much polymorphism.

\subsection{Specialize}

Replace uses of parameterized datatypes with versions specialized to specific parameters.  As for Unpoly, this phase will not be effective enough in the presence of polymorphic recursion or other fancy uses of impredicative polymorphism.

\subsection{Shake}

Here the compiler repeats the earlier Shake phase.

\subsection{Monoize}

Programs are translated to a new intermediate language without polymorphism or non-$\mt{Type}$ constructors.  Error messages may pop up here if earlier phases failed to remove such features.

This is the stage at which concrete names are generated for cookies, tables, and sequences.  They are named following the same convention as for links and actions, based on module path information saved from earlier stages.  Table and sequence names separate path elements with underscores instead of slashes, and they are prefixed by \texttt{uw\_}.
\subsection{MonoOpt}

Simple algebraic laws are applied to simplify the program, focusing especially on efficient imperative generation of HTML pages.

\subsection{MonoUntangle}

Unnecessary mutual recursion is broken up again.

\subsection{MonoReduce}

Equivalents of the definitional equality rules are applied to simplify programs, with inlining again playing a major role.

\subsection{MonoShake, MonoOpt}

Unneeded declarations are removed, and basic optimizations are repeated.

\subsection{Fuse}

The compiler tries to simplify calls to recursive functions whose results are immediately written as page output.  The write action is pushed inside the function definitions to avoid allocation of intermediate results.

\subsection{MonoUntangle, MonoShake}

Fuse often creates more opportunities to remove spurious mutual recursion.

\subsection{Pathcheck}

The compiler checks that no link or action name has been used more than once.

\subsection{Cjrize}

The program is translated to what is more or less a subset of C.  If any use of functions as data remains at this point, the compiler will complain.

\subsection{C Compilation and Linking}

The output of the last phase is pretty-printed as C source code and passed to GCC.


\end{document}